%----------------------------------------------------------------------------------------
%    PAGE ADJUSTMENTS
%----------------------------------------------------------------------------------------

\documentclass[12pt,a4paper]{scrartcl}            % Article 12pt font for a4 paper while hiding links
\usepackage[margin=1.25in]{geometry}                          % Required to adjust margins

%----------------------------------------------------------------------------------------
%    TYPE SETTING PACKAGES
%----------------------------------------------------------------------------------------

\usepackage[english]{babel}                                % English language/hyphenation 
\usepackage[utf8x]{inputenc}                               % Accept different input encodings
\usepackage{amsmath,amsfonts,amsthm,amssymb}               % Math packages to use equations
\usepackage{siunitx}                                       % Scientific units and numbering
\usepackage[usenames,dvipsnames,svgnames,table]{xcolor}    % Set color of text/background
\linespread{1}                                           % Default line spacing size
\usepackage{microtype}                                     % Improves spacing in the document
\usepackage{setspace}                                      % Set line spacing dynamically
\usepackage{tocloft}                                       % List adjustments including ToC
\DeclareMathOperator*{\argmin}{arg\!\min}
\definecolor{purple}{HTML}{7A68A6}
\definecolor{blue}{HTML}{2020A1}
\definecolor{red}{HTML}{A60628}
%----------------------------------------------------------------------------------------
%    FIGURES
%----------------------------------------------------------------------------------------

\usepackage{graphicx}                                      % Required for the inclusion of images
\graphicspath{{./Pictures/}}                               % Specifies picture directory
\usepackage{float}                                         % Allows putting an [H] in \begin{figure}
\usepackage{wrapfig}                                       % Allows in-line images

\usepackage[colorlinks=true, citecolor=blue]{hyperref}     % References
\usepackage{cleveref}                                      % Better References
%\crefname{lstlisting}{listing}{listings}
%\Crefname{lstlisting}{Listing}{Listings}
\crefname{figure}{figure}{figures}
\Crefname{figure}{Figure}{Figures}

%----------------------------------------------------------------------------------------
%    INCLUDE CODE
%----------------------------------------------------------------------------------------

\usepackage{listings}                                      % Package so code looks pretty
\lstset{
language=Python,                                           % Choose the language
basicstyle=\footnotesize,                                  % The size of the fonts used
numbers=left,                                              % Where to put the line-numbers
numberstyle=\footnotesize,                                 % The size of the line-numbers
stepnumber=1,                                              % The step line-numbers
numbersep=5pt,                                             % How far the line-numbers are from the code
backgroundcolor=\color{white},                             % Choose the background color
showspaces=false,                                          % Show spaces adding partiular underscores
showstringspaces=false,                                    % Underline spaces within strings
showtabs=false,                                            % Show tabs within strings adding particular underscores
frame=single,                                              % Adds a frame around the code
tabsize=2,                                                 % Sets default tabsize to 2 spaces
captionpos=b,                                              % Sets the caption-position to bottom
breaklines=true,                                           % Sets automatic line breaking
breakatwhitespace=false,                                   % Sets if automatic breaks should only happen at whitespace
escapeinside={\%*}{*)}                                     % If you want to add a comment within your code
}

\usepackage[T1]{fontenc}
\usepackage{inconsolata}


%----------------------------------------------------------------------------------------
%    EXTRAS
%----------------------------------------------------------------------------------------

\usepackage{attachfile}                                    % Attach files to your document
\usepackage{fancyhdr}                                      % Fancy Header
\usepackage{natbib}

\begin{document}

%----------------------------------------------------------------------------------------
%    COMMANDS
%----------------------------------------------------------------------------------------

\renewcommand*\thesection{\arabic{section}}                % Renew section numbers
\renewcommand{\labelenumi}{\alph{enumi}.}                  % Section ordered numbering
\let\oldvec\vec                                            % Save the old vector style
\renewcommand{\vec}[1]{\oldvec{\mathbf{#1}}}               % Set vectors to look like vectors

\renewcommand{\contentsname}{Table of Contents}            % Make ToC actually say ToC
\addtocontents{toc}{~\hfill\textbf{Page}\par}              % Add 'page' to top of ToC
\renewcommand{\cftsecleader}{\cftdotfill{\cftdotsep}}      % Makes dots leading up to page number
\setcounter{tocdepth}{3}                                   % Depth of ToC
\setcounter{lofdepth}{3}                                   % Depth of LoF

\pagestyle{plain}                                          % Fancy page style for headers
% \setlength{\headheight}{15pt}                              % Change header hieght
% \fancyhead[L,LO]{\fontsize{8}{10} \selectfont \firstmark}  % Adds header to left with section name
% \fancyhead[R,RO]{\fontsize{8}{10} \selectfont Right}       % Adds header to right
\definecolor{grey}{HTML}{cccccc}                           % The next 4 lines modifies the header (color)
\renewcommand{\headrulewidth}{1px}
\renewcommand{\headrule}{{\color{grey}%
\hrule width\headwidth height\headrulewidth%
\vskip-\headrulewidth}}

\numberwithin{equation}{section}                           % Number equations within sections
\numberwithin{figure}{section}                             % Number figures within sections
\numberwithin{table}{section}                              % Number tables within sections
\numberwithin{lstlisting}{section}                         % Number listings within sections

% \renewcommand{\sfdefault}{phv}                             % Change default font
% \renewcommand{\familydefault}{\sfdefault}                  % Use default font everywhere
\newcommand{\tvect}[2]{%
  \ensuremath{\Bigl(\negthinspace\begin{smallmatrix}#1\\#2\end{smallmatrix}\Bigr)}}

\makeatletter
\def\blfootnote{\xdef\@thefnmark{}\@footnotetext}
\makeatother


%----------------------------------------------------------------------------------------
%    TITLE PAGE
%----------------------------------------------------------------------------------------

% \begin{titlepage}
\title{Monetary Policy with the Zero-Lower Bound and Wage Rigidities}
\subtitle{Research Proposal for Second-Year Paper}
\author{Tom Augspurger}                               % via Seth Miers

% \vspace*{\fill}                                            % Center title page vertically

% \newcommand{\HRule}{\rule{\linewidth}{0.3mm}}              % Defines horizontal lines

% \center                                                    % Center everything on the page

% \textsc{\LARGE Research Proposal}\[1.5cm]                % First heading

% \HRule \[0.4cm]
% { \huge \bfseries Title}\[0.4cm]                     % Title of document
% \HRule \[1.5cm]

%\begin{minipage}{0.4\textwidth}
%\begin{flushleft} \large
%\emph{Author:}\
%Tom \textsc{Augspurger}                                    % Name
%\end{flushleft}
%\end{minipage}
%~
%\begin{minipage}{0.4\textwidth}
%\begin{flushright} \large
%\emph{Professor:} \
%Dr. James \textsc{Smith}                                   % Professor's Name
%\end{flushright}
%\end{minipage}\[4cm]
\maketitle
% \center{{\today}}                                                  % Date, change the \today to be precise

%\includegraphics{Logo}\[1cm]                             % Include a departmentw/university logo

% \vspace*{\fill}                                            % Fill the rest of the page with whitespace

% \end{titlepage}

% \phantomsection


%----------------------------------------------------------------------------------------
%    CONTENT
%----------------------------------------------------------------------------------------

\begin{abstract}
For my second-year paper I plan to investigate the effects downward nominal wage rigidities and the zero-lower-bound on nominal interest rates have on optimal monetary policy.
The downward-nominal-wage rigidity literature has emphasized the importance of idiosyncratic shocks in generating effects on aggregate output.
To date, work on the zero-lower bound has not incorporated this point.
My main contribution will be to combine the finding that idiosyncratic labor shocks matter with the research on optimal monetary policy in a world with the zero-lower bound.
It is possible that downward-nominal-wage rigidities will raise the optimal inflation target set by the central bank.
Furthermore, endogenous wage rigidities in the upward direction may slow wage growth and inflation when exiting the zero-lower bound, which may mitigate expansionary actions taken by the central bank.
\end{abstract}

\blfootnote{Thanks to Martin Gervais, Alice Schoonbroodt, and Nicolas Ziebarth for agreeing to advise me on this project.  All remaining errors are mine, etc.}
\newpage
%----------------------------------------------------------------------------------------
\section{Introduction}
\label{sec:introduction}

The zero-lower-bound on nominal interest rates has a long history in economics. It has seen renewed attention since Japan ran up against it in the 1990s and even more so since central banks in advanced countries cut their policy rates nearly to zero with the recent recession.
Hitting the zero-lower bound matters mainly because it deepens recessions and slows recovery (\cite{williams_2009} gives an estimate of the cost).
% The zero-lower bound may affect many of our predictions, policy recommendations, and modeling choices;\footnote{\cite{wieland_2013} and \cite{curdia_woodford_2010} on predictions and policy recommendations and \cite{fv_gordon_gq_rr_2012} on modeling at the zero-lower bound.}
There are many issues to be considered with the zero-lower bound, but I will focus mainly on how monetary policy should act when the zero-lower bound may bind;
specifically, should central banks raise their inflation targets due to the zero-lower bound?
If hitting the zero-lower bound is costly, we may benefit by targeting higher inflation and hitting it less often.
However, this benefit has to be measured against the costs of higher inflation (mainly price dispersion and inefficient allocation of resources).
% A world where the zero lower bound binds is likely also to be a world where inflation is low (cite Great Depression, Japan; to a smaller extent today).  The downward wage rigidity is likely to bind when inflation is unexpectedly low.

%----------------------------------------------------------------------------------------
% Wage rigidities intro.

% There is much evidence for wage rigidities\footnote{For example, \cite{dickens_et_al_2006}}.
% % The effects are perhaps harder to pin down than the existence.
A separate literature has investigated the phenomenon of downward wage rigidities.
The wage rigidity literature has two principal components: an empirical side concerned with the evidence and possible reasons for wage rigidities at the micro level and another side concerned with the economic implications of wage rigidities in aggregate.\footnote{For the micro level evidence you can start with \cite{dickens_et_al_2006}, \cite{akerlof_dickens_perry_1996}, or \cite{card_hyslop_1997}. \cite{erceg_henderson_levin_1999} is an example of the the macroeconomic effects.  \cite{elsby_2009} finds weaker macroeconomic effects. \cite{chari_kehoe_macgratten_2009} would disagree with the whole endeavor.}
I will mostly ignore the micro side---the reasons why wages appear to be rigid---and instead focus on how downward-nominal-wage rigidity affects monetary policy.

One interesting result from prior research on wage rigidities is that idiosyncratic shocks to labor are important for generating aggregate effects on output or employment.\footnote{This is stressed in \cite{elsby_2009}, \cite{benigno_ricci_2011}, and \cite{daly_hobijn_2013}.}
\cite{daly_hobijn_2013} \emph{assert} that ``models need to include heterogeneous shocks to the labor supply or productivity of workers. Such shocks are necessary to make nominal wage cuts desirable when the economy is in steady state.''
I will describe my setup more fully later, but for now bear in mind that these idiosyncratic shocks are something of a linchpin to my story.
While we can see how idiosyncratic shocks will distort relative prices, at this time I do not have a compelling explanation for how much these shock matter or what their worldly counterpart is.\footnote{
Interestingly, \cite{golosov_lucas_2007} introduced similar idiosyncratic shocks in a model with menu costs.
Their motivation was to reconcile the level of aggregate inflation with individual price changes;
aggregate shocks alone could not produce the observed dynamics.
}
At the very least I will be able to benchmark my model with idiosyncratic shocks against a model with only aggregate shocks to quantify the difference.

% I reiterate that idiosyncratic shocks are important for downward nominal wage rigidities to generate effects at the aggregate level.

%----------------------------------------------------------------------------------------
% review of coibion
Some attempts have been made to combine downward-nominal-wage rigidity with the zero-lower bound.
Most notably \cite{coibon_gorodnichenko_wieland_2012} included downward-nominal-wage rigidity as they investigated optimal monetary policy with the zero-lower bound.
My approach will be similar to theirs, though with the (perhaps) crucial difference that I will use idiosyncratic shocks to labor.
If the research on downward-nominal-wage rigidity is to be believed, this difference will prove to be important.
Contrary to their expectations, \cite{coibon_gorodnichenko_wieland_2012} found that downward-nominal-wage rigidities were actually welfare improving.
In their model wage rigidities reduced the volatility of marginal costs and output, and so reduced the variance of inflation.
The wage rigidities were also a bulwark against deflation, so the central bank could target a lower inflation level while maintaining the same probability of the zero-lower bound binding.
It is not immediately clear if these beneficial effects of downward-nominal-wage rigidity will still be present when idiosyncratic shocks are considered.
%----------------------------------------------------------------------------------------
\section{My Motivation}
\label{sec:my_motivation}

I am interested in modeling both the zero-lower bound and downward-nominal-wage rigidity together since we might expect them to become binding in the same environment.
The zero-lower bound is usually characterized by low or even negative inflation.
And a world with unexpectedly low inflation is a world where downward-nominal-wage rigidities should be most likely to bind as well.

Another reason to investigate the zero-lower bound and downward-nominal-wage rigidity simultaneously is that their costs may both be reduced by a higher rate of inflation.
New Keynesian models typically recommend an inflation rate of zero or even deflation equal to the real rate of interest.\footnote{\cite{schmitt-grohe_uribe_2010} gives a good summary.  Famously, Oliver Blanchard suggested the possibility of higher inflation targets in \cite{blanchard_dellariccia_and_mauro}.}
Correcting for nominal wage rigidities has been offered as a benefit of inflation in the past, perhaps first by \cite{tobin_1972};
and, as mentioned in the introduction, positive inflation implies a lower probability of hitting the zero lower bound.
We should then expect a higher optimal rate of inflation when the zero-lower bound and downward-nominal-wage rigidities are included.

% This needs some work: given inflation level, DNWR implies lower w' which implies a lower level of inflation?
Furthermore, it's possible that downward-nominal-wage rigidities could dampen growth in the recovery from a zero-lower-bound event.
Households will choose lower wages today (relative to the flexible-wage case) because they anticipate future dates where they would like to lower wages but cannot (fear of slipping back into the zero-lower bound, say).
Marginal costs and inflation will rise by less, tempering the fall in the real interest rate.
This is the endogenous \emph{upward}-wage rigidity process described in \cite{elsby_2009}.
Placed inside a model with the zero-lower bound, this upward rigidity takes on new importance:
outside of the zero-lower bound the central bank can achieve its target by lowering the nominal interest rate;
at the zero-lower bound the central bank does not have that option.
Of course, a model is needed to discuss the exact nature of the interaction and to quantify the various effects.

%----------------------------------------------------------------------------------------
\section{The Model}
\label{sec:the_model}

More concretely, I will use a fairly standard New Keynesian model where the representative household supplies differentiated labor and consumes the final product.
Individuals in the household (the different labor types, indexed on $[0, 1]$) will be subject to idiosyncratic shocks each period; \cite{daly_hobijn_2013} use preference shocks to the disutility from labor and I will likely do the same, though I am also considering productivity shocks as in \cite{fagan_messina_2009}. 
With preference shocks, the household's utility function will be something along the lines of
\begin{equation*}
    E_{t_0} \sum_{t=t_0}^{\infty} \beta^t \left\{\ln{C_t} - \frac{\eta}{\eta + 1} \int_{0}^{1} Z_t(i) N_t(i)^{\frac{\eta + 1}{\eta}} di \right\}
\end{equation*}
where $C_t$ is the household's consumption, $N_t(i)$ is individual $i$'s labor contribution, $\eta > 1$ is the Frisch labor supply elasticity, and $Z_t(i)$ is a stochastic preference shock to labor of type $i$.\footnote{I don't want to commit to a specification too early---tractability will likely put huge importance on the exact forms I choose---but the shock Z will probably be log-normally distributed with independence across time.}
This will be maximized subject to the period budget constraint:

\begin{equation*}
    C_t + \frac{S_t}{P_t} \leq \int_{0}^{1} \left( N_t(i) \frac{W_t(i)}{P_t}\right) di + S_{t-1}\frac{R_{t-1}}{P_t} + T_t
\end{equation*}
Here, $S_t$ is the stock of one-period bonds carried over to tomorrow, $R_t$ is the gross nominal interest rate, $P_t$ is the price of the final good, $W_t(i)$ is the nominal wage paid to individual $i$, and $T_t$ is a real transfer of profits from the firms to the household.

Like in \cite{erceg_henderson_levin_1999} individuals of the household can share consumption risk but make wage decisions individually.
Given this wage, each individual supplies labor as demanded.  Additionally, we will construct a labor aggregator who chooses the labor types in the same proportion as the individual firms described below.  It demands a labor index $N_t$ defined as:

\begin{equation*}
    N_t = \left[ \int_{0}^{1} N_t(h)^{ \frac{\eta - 1}{\eta} } dh \right]^{ \frac{\eta}{\eta - 1} }
\end{equation*}
where the integration is over the labor types. The aggregator's total demand for each type of labor equals the sum of the intermediate firms' demand for that type.
In the usual manner, this aggregator minimizes the input costs required to produce a given level of output by choosing inputs $N_t(h)$ given wages $W_t(h)$.  This leads to the aggregate wage index $W_t$:

\begin{equation*}
    W_t \equiv \left[ \int_{0}^{1} W_t(h)^{ 1 - \eta }  \right]^{ \frac{1}{1 - \eta} }
\end{equation*}
Finally, this implies that the total demand for labor services from individual $h$ is:

\begin{equation*}
    N_t(h) = \left[ \frac{W_t}{W_t(h)}  \right]^{\eta}N_t
\end{equation*}


The production side of the economy will be characterized by a continuum of monopolistically competitive  intermediate goods producers and a perfectly competitive final good producer.
The intermediate producers are also indexed on $[0, 1]$.
As before, we'll construct an aggregator who combines the intermediate products $Y_t(f)$ into an aggregate index $Y_t$.
The aggregator's choices reflect those of the household:
\begin{equation*}
    Y_t = \left[\int_{0}^{1} Y_t(f)^{ \frac{\theta - 1}{\theta}} df \right]^{ \frac{\theta}{\theta - 1} }
\end{equation*}
where $\theta > 1$ is the elasticity of substitution among types of intermediate goods.
This yields the aggregate price index:

\begin{equation*}
    P_t \equiv \left[\int_{0}^{1} P_t(f)^{1 - \theta} df \right]^{\frac{1}{1 - \theta}}
\end{equation*}
where $P_t(f)$ is the price for intermediate product $f$.
We can also get the total demand for that firm's product:

\begin{equation*}
    Y_t(f) = \left[ \frac{P_t}{P_t(f)}  \right]^{\theta} Y_t
\end{equation*}

Meanwhile each intermediate good $f$ is produced according to:

\begin{equation*}
    Y_t(f) = A_t N_t(f)
\end{equation*}
where $A_t$ is the (stochastic) aggregate technology level and $N_t(f)$ is the quantity of the aggregate labor index $N_t$ chosen by that firm.
The intermediate goods producers will likely face some sort of pricing friction as in \cite{calvo_1983}.

I haven't settled on a specific form for the downward nominal wage rigidities yet.
\cite{daly_hobijn_2013} use a Calvo-type friction where only some fraction of workers are allowed to change their wages downward.
\cite{coibon_gorodnichenko_wieland_2012} mandate that the wage today is the maximum of the wage yesterday and the desired wage today.
Alternatively, \cite{benigno_ricci_2011} restrict the workers' path of wages to the space of (weakly) increasing sequences.
Regardless of the specific form I chose, this will be a good area for sensitivity analysis.

The central bank will follow a Taylor rule that is constrained by the zero-lower bound.
I should then be able log-linearize most variables around the steady-state, which will facilitate calibration and solution for the optimal inflation rate.
Some extra care will be needed for potential nonlinearities at the zero-lower bound (\citealt{fv_gordon_gq_rr_2012}) and the heterogeneity introduced by the labor shocks.
Welfare will be measured by the household's utility function and the loss will be relative to the benchmark case of no zero-lower bound and no downward-nominal-wage rigidity.

I will use the standard National Income and Product Accounts for most of the aggregate data.
Using data from the Current Population Survey, I am constructing a panel dataset to measure the distribution of wage changes at the monthly or quarterly frequency.
It may be interesting to track the distribution of wage changes over time in my calibration.
I suspect that this aspect will provide some technical difficulties, but hopefully nothing too intractable.

%----------------------------------------------------------------------------------------
% Why does heterogeneity matter?
% From Daly and Hobijn: ```First, for downward nominal wage rigidities to have a meaningful impact on unemployment in standard macroeconomic models, the models need to include heterogeneous shocks to the labor supply or productivity of workers. Such shocks are necessary to make nominal wage cuts desirable when the economy is in steady state.'''

% and also

% ```The existence of the idiosyncratic shocks is thus a central part of this argument. Without them, downward nominal wage rigidities are unlikely to cause a substantial distortion in equilibrium. In fact, most studies of downward nominal wage rigidities that do not include such shocks find that these rigidities have a very small effect on labor market allocations and welfare.
% '''

% Torben M. Andersen (2001), presenting a static model which can be solved in a closed form, and V. Bhaskar (2002

% \newpage
\bibliographystyle{apa}
\bibliography{citations}

\end{document}
