%----------------------------------------------------------------------------------------
%    PAGE ADJUSTMENTS
%----------------------------------------------------------------------------------------

\documentclass[12pt,a4paper]{scrartcl}            % Article 12pt font for a4 paper while hiding links
\usepackage[margin=1.25in]{geometry}                          % Required to adjust margins

%----------------------------------------------------------------------------------------
%    TYPE SETTING PACKAGES
%----------------------------------------------------------------------------------------

\usepackage[english]{babel}                                % English language/hyphenation
\usepackage[utf8x]{inputenc}                               % Accept different input encodings
\usepackage{amsmath,amsfonts,amsthm,amssymb}               % Math packages to use equations
\usepackage{siunitx}                                       % Scientific units and numbering
\usepackage[usenames,dvipsnames,svgnames,table]{xcolor}    % Set color of text/background
\linespread{1}                                           % Default line spacing size
\usepackage{microtype}                                     % Improves spacing in the document
\usepackage{setspace}                                      % Set line spacing dynamically
\usepackage{tocloft}                                       % List adjustments including ToC
\DeclareMathOperator*{\argmin}{arg\!\min}
\definecolor{purple}{HTML}{7A68A6}
\definecolor{blue}{HTML}{2020A1}
\definecolor{red}{HTML}{A60628}
%----------------------------------------------------------------------------------------
%    FIGURES
%----------------------------------------------------------------------------------------

\usepackage{graphicx}                                      % Required for the inclusion of images
\graphicspath{{./Pictures/}}                               % Specifies picture directory
\usepackage{float}                                         % Allows putting an [H] in \begin{figure}
\usepackage{wrapfig}                                       % Allows in-line images

\usepackage[colorlinks=true, citecolor=blue]{hyperref}     % References
\usepackage{cleveref}                                      % Better References
%\crefname{lstlisting}{listing}{listings}
%\Crefname{lstlisting}{Listing}{Listings}
\crefname{figure}{figure}{figures}
\Crefname{figure}{Figure}{Figures}

%----------------------------------------------------------------------------------------
%    INCLUDE CODE
%----------------------------------------------------------------------------------------

\usepackage{listings}                                      % Package so code looks pretty
\lstset{
language=Python,                                           % Choose the language
basicstyle=\footnotesize,                                  % The size of the fonts used
numbers=left,                                              % Where to put the line-numbers
numberstyle=\footnotesize,                                 % The size of the line-numbers
stepnumber=1,                                              % The step line-numbers
numbersep=5pt,                                             % How far the line-numbers are from the code
backgroundcolor=\color{white},                             % Choose the background color
showspaces=false,                                          % Show spaces adding partiular underscores
showstringspaces=false,                                    % Underline spaces within strings
showtabs=false,                                            % Show tabs within strings adding particular underscores
frame=single,                                              % Adds a frame around the code
tabsize=2,                                                 % Sets default tabsize to 2 spaces
captionpos=b,                                              % Sets the caption-position to bottom
breaklines=true,                                           % Sets automatic line breaking
breakatwhitespace=false,                                   % Sets if automatic breaks should only happen at whitespace
escapeinside={\%*}{*)}                                     % If you want to add a comment within your code
}

\usepackage[T1]{fontenc}
\usepackage{inconsolata}


%----------------------------------------------------------------------------------------
%    EXTRAS
%----------------------------------------------------------------------------------------

\usepackage{attachfile}                                    % Attach files to your document
\usepackage{fancyhdr}                                      % Fancy Header
\usepackage{natbib}

\begin{document}

%----------------------------------------------------------------------------------------
%    COMMANDS
%----------------------------------------------------------------------------------------

\renewcommand*\thesection{\arabic{section}}                % Renew section numbers
\renewcommand{\labelenumi}{\alph{enumi}.}                  % Section ordered numbering
\let\oldvec\vec                                            % Save the old vector style
\renewcommand{\vec}[1]{\oldvec{\mathbf{#1}}}               % Set vectors to look like vectors

\renewcommand{\contentsname}{Table of Contents}            % Make ToC actually say ToC
\addtocontents{toc}{~\hfill\textbf{Page}\par}              % Add 'page' to top of ToC
\renewcommand{\cftsecleader}{\cftdotfill{\cftdotsep}}      % Makes dots leading up to page number
\setcounter{tocdepth}{3}                                   % Depth of ToC
\setcounter{lofdepth}{3}                                   % Depth of LoF

\pagestyle{plain}                                          % Fancy page style for headers
% \setlength{\headheight}{15pt}                              % Change header hieght
% \fancyhead[L,LO]{\fontsize{8}{10} \selectfont \firstmark}  % Adds header to left with section name
% \fancyhead[R,RO]{\fontsize{8}{10} \selectfont Right}       % Adds header to right
\definecolor{grey}{HTML}{cccccc}                           % The next 4 lines modifies the header (color)
\renewcommand{\headrulewidth}{1px}
\renewcommand{\headrule}{{\color{grey}%
\hrule width\headwidth height\headrulewidth%
\vskip-\headrulewidth}}

\numberwithin{equation}{section}                           % Number equations within sections
\numberwithin{figure}{section}                             % Number figures within sections
\numberwithin{table}{section}                              % Number tables within sections
\numberwithin{lstlisting}{section}                         % Number listings within sections

% \renewcommand{\sfdefault}{phv}                             % Change default font
% \renewcommand{\familydefault}{\sfdefault}                  % Use default font everywhere
\newcommand{\tvect}[2]{%
  \ensuremath{\Bigl(\negthinspace\begin{smallmatrix}#1\\#2\end{smallmatrix}\Bigr)}}

\makeatletter
\def\blfootnote{\xdef\@thefnmark{}\@footnotetext}
\makeatother


%----------------------------------------------------------------------------------------
%    TITLE PAGE
%----------------------------------------------------------------------------------------

% \begin{titlepage}
\title{Monetary Policy with the Zero-Lower Bound and Wage Rigidities}
\subtitle{Research Proposal for Second-Year Paper}
\author{Tom Augspurger}                               % via Seth Miers

% \vspace*{\fill}                                            % Center title page vertically

% \newcommand{\HRule}{\rule{\linewidth}{0.3mm}}              % Defines horizontal lines

% \center                                                    % Center everything on the page

% \textsc{\LARGE Research Proposal}\[1.5cm]                % First heading

% \HRule \[0.4cm]
% { \huge \bfseries Title}\[0.4cm]                     % Title of document
% \HRule \[1.5cm]

%\begin{minipage}{0.4\textwidth}
%\begin{flushleft} \large
%\emph{Author:}\
%Tom \textsc{Augspurger}                                    % Name
%\end{flushleft}
%\end{minipage}
%~
%\begin{minipage}{0.4\textwidth}
%\begin{flushright} \large
%\emph{Professor:} \
%Dr. James \textsc{Smith}                                   % Professor's Name
%\end{flushright}
%\end{minipage}\[4cm]
\maketitle
% \center{{\today}}                                                  % Date, change the \today to be precise

%\includegraphics{Logo}\[1cm]                             % Include a departmentw/university logo

% \vspace*{\fill}                                            % Fill the rest of the page with whitespace

% \end{titlepage}

% \phantomsection


%----------------------------------------------------------------------------------------
%    CONTENT
%----------------------------------------------------------------------------------------

\begin{abstract}
I use a simplified New Keynesian model to explore how downward nominal wage rigidities affect welfare and optimal monetary policy.

\end{abstract}

\section{Introduction}
\label{sec:introduction}

One interesting result from prior research on wage rigidities is that idiosyncratic shocks to labor are important for generating aggregate effects on output or employment.\footnote{
This is stressed in \cite{elsby_2009}, \cite{benigno_ricci_2011}, and \cite{daly_hobijn_2013}.}
\cite{daly_hobijn_2013} \emph{assert} that ``models need to include heterogeneous shocks to the labor supply or productivity of workers. Such shocks are necessary to make nominal wage cuts desirable when the economy is in steady state.''
I will describe my setup more fully later, but for now bear in mind that these idiosyncratic shocks are something of a linchpin to my story.
While we can see how idiosyncratic shocks will distort relative prices, at this time I do not have a compelling explanation for how much these shock matter or what their worldly counterpart is.\footnote{
Interestingly, \cite{golosov_lucas_2007} introduced similar idiosyncratic shocks in a model with menu costs.
Their motivation was to reconcile the level of aggregate inflation with individual price changes;
aggregate shocks alone could not produce the observed dynamics.
}
At the very least I will be able to benchmark my model with idiosyncratic shocks against a model with only aggregate shocks to quantify the difference.

% I reiterate that idiosyncratic shocks are important for downward nominal wage rigidities to generate effects at the aggregate level.

%----------------------------------------------------------------------------------------
% review of coibion
Some attempts have been made to combine downward-nominal-wage rigidity with the zero-lower bound.
Most notably \cite{coibon_gorodnichenko_wieland_2012} included downward-nominal-wage rigidity as they investigated optimal monetary policy with the zero-lower bound.
My approach will be similar to theirs, though with the (perhaps) crucial difference that I will use idiosyncratic shocks to labor.
If the research on downward-nominal-wage rigidity is to be believed, this difference will prove to be important.
Contrary to their expectations, \cite{coibon_gorodnichenko_wieland_2012} found that downward-nominal-wage rigidities were actually welfare improving.
In their model wage rigidities reduced the volatility of marginal costs and output, and so reduced the variance of inflation.
The wage rigidities were also a bulwark against deflation, so the central bank could target a lower inflation level while maintaining the same probability of the zero-lower bound binding.
It is not immediately clear if these beneficial effects of downward-nominal-wage rigidity will still be present when idiosyncratic shocks are considered.

%----------------------------------------------------------------------------------------
\section{Literature Review}
\label{sec:literature_review}

Almost invariably starts with \cite{keynes2010general} and his discussion of downward nominal wage rigidity.
Somewhat more recently, \cite{akerlof_dickens_perry_1996} 

\section{Model}

\section{The Model}
\label{sec:the_model}

I follow \cite{daly_hobijn_2013} almost exactly.  We'll set up a model with a continuum of individuals who sell their labor services to a perfectly competitive final goods producers.  Individuals receive idiosyncratic shocks to their labor preference. The interaction of these shocks with an exogenous impediment to \emph{lowering} wages is the area of research.

\subsection{Production}
\label{sub:production}


Aggregate output is linear in labor $L_t$.

\begin{equation} \label{eq:agg_output}
    f(L_t) = L_t = Y_t.
\end{equation}

A simplistic specification to be sure.  \textcolor{red}{My results are sensitive to different assumptions about productivity growth...}

Aggregate labor \(L_t\) is defined to be the integral over each individual's labor \(L_t(i)\):

\begin{equation} \label{eq:agg_labor}
    L_t = \left[ \int_0^1 L_t(i)^{\frac{\eta - 1}{\eta}} \mathrm{d}i \right]^{\frac{\eta}{\eta - 1}}.
\end{equation}

where $\eta$ is the usual elasticity of substitution.

To simplify the aggregation process, I assume that there is a single final \textcolor{red}{producer / final good} who operates in a perfectly competitive market.
This final good producer's problem is to maximize profits (which will be zero in equilibrium):

\begin{equation} \label{eq:firms_problem}
    \max_{L_t(i)} P_t L_t - \int_0^1 W_t(i)L_t(i)
\end{equation}

taking individual wages $W_t(i)$ and the aggregate price level $P_t$ as given.

We can write the aggregate wage index $W_t$ as

\begin{equation} \label{eq:wage_index}
    W_t = \left[\int_{0}^{1}\frac{1}{W_t(i)}^{\eta - 1} \mathrm{d}i \right]^{-\frac{1}{\eta - 1}}.
\end{equation}

Thanks to the assumption of perfect competition, we know that the aggregate price level must equal the aggregate wage index, $P_t = W_t$.  This implies that the real aggregate wage index is equal to one and so

\begin{equation} \label{eq:real_wage}
    1 = \frac{W_t}{P_t} = \left[\int_{0}^{1} \left( \frac{P_t}{W_t(i)} \right)^{\eta - 1} \mathrm{d}i \right]^{-\frac{1}{\eta - 1} } = \left[\int_{0}^{1} \! w_{it}^{1 - \eta} \mathrm{d}i \right]^{ \frac{1}{1 - \eta} }
\end{equation}

where $w_{it}$ is the real wage charged by individual $i$.

The interesting choices in the model are made by the individuals, to whom we turn next.

\subsection{The Household}
\label{sub:The Household}

There is a representative household made up of a continuum of individuals.
As in \cite{erceg_henderson_levin_1999} individuals can share consumption risk perfectly, but make wage setting decisions on their own.
Once an individual has chosen a wage, he then supplies as much labor as is demanded by the firm.

The household's utility function is

\begin{equation} \label{eq:utility}
    \sum_{t=0}^{\infty} \beta^t \left\{\ln Y_t - \frac{\gamma}{\gamma + 1} \int_{0}^{1} Z_t(i)L_t(i)^{\frac{\gamma + 1}{\gamma}}\mathrm{d}i\right\}
\end{equation}

where $\gamma > 0$ is the elasticity of labor supply and $\beta \in (0, 1)$ is the discount factor.
The import bit of this utility function is that the labor preference shocks, $Z_t(i)$ are specific to each individual.

The utility function is maximized subject to the budget constraint, common across households,

\begin{equation}
    \label{eq:budget}
    A_t = \left(1 + i_{t-1}\right)A_{t-1} + \int_{0}^{1} W_t(i)L_t(i)\mathrm{d}i
\end{equation}

where $A_t$ is the level of nominal assets held by the household at time $t$ and $i$ is the nominal interest rate.
As mentioned earlier each individual sets his own wage and, given that wage, works what is demanded by the firm according to the labor demand curve

\begin{equation}
    \label{eq:labor_demand}
    L_t(i) = \left( \frac{W_t(i)}{W_t} \right)^{-\eta}L_t = (w_{it})^{-eta}L_t
\end{equation}

So, all else equal, individuals will work more the lower their own real wage or the higher aggregate output.

The household takes the paths of aggregate wages, output, and final goods price $\left\{W_t, L_t, P_t \right\}_{t=0}^{\infty}$ as given.

I'll assume that the idiosyncratic shocks to labor preference come from a lognormal distribution, i.i.d. across individuals and time:

\begin{equation}
    \label{eq:shock_dist}
    \ln(Z) \sim \mathcal{N}\left(-\frac{\sigma^2}{2}, \sigma\right)
\end{equation}

so that $\mathbb{E}(Z) = 1$.
This particular specification is chosen mainly out of convenience.

Finally, to the purpose at hand, I assume that nominal wages are rigid in the downward direction.
In the spirit of \cite{calvo_1983}, every period a randomly selected group of individuals of measure $\lambda$ are not allowed to \emph{lower} their nominal wages.
The other $(1 - \lambda)$ individuals are free to choose whatever wage they please.
\subsection{Wage Setting}
\label{sub:wage_setting}

We'll consider the flexible case ($\lambda = 0$) first.

Letting
\begin{equation}
    \label{eq:labor_part}
    \Omega( w_t(i); Z_t(i), L_t ) = w_{it}^{1 - \eta} - \frac{\gamma}{\gamma + 1}Z_t(i)\left[ w_{it}^{-\eta}L_t \right]^{\frac{\gamma}{1 + \gamma}}
\end{equation}

be the labor-related terms of the household's problem, the optimal wage in the flexible can can be attained by maximizing

\begin{equation}
    \label{eq:labor_opt}
    \mathbb{E}_t\left[\sum_{t=0}^{\infty}\beta^t \Omega( w_t(i); Z_t(i), L_t ) \right].
\end{equation}

Solving this gives the optimal real wage schedule

\begin{equation}
    \label{eq:flex}
    \hat{w}(Z_t(i); L_t) = \hat{w}_{it} = \left( \frac{\eta}{\eta - 1} \right)^{\frac{\gamma}{\eta + \gamma}}\left( Z_t(i) \right)^{\frac{\gamma}{\eta + \gamma}} L_t^{\frac{\gamma + 1}{\gamma + \eta}}
\end{equation}

For the rigid case, $\lambda > 0$, our approach is slightly different
Each day, our individuals wake up with their wage from yesterday.
They receive a labor preference shock, $Z_t(i)$, and find out whether they are allowed to choose any nominal wage, or just a wage greater than yesterday's.
They then choose the wage for today to optimize today's utility plus the future value of holding that wage.
As a value function, holding a real wage $w$:

\begin{multline}
    \label{eq:value_function}
    V_t(w) = (1 - \lambda) \int_{0}^{\infty} \max_{w_{it} \geq 0} \left\{ \Omega( w_t(i); Z_t(i), L_t ) + \beta V_{t+1}\left( \frac{w_{it}}{(1 + \pi_{t+1})} \right) \right\} \mathrm{d}F(Z_t(i)) \\
                + \lambda  \int_{0}^{\infty} \max_{w_{it} \geq w} \left\{ \Omega( w_t(i); Z_t(i), L_t ) + \beta V_{t+1}\left( \frac{w_{it}}{(1 + \pi_{t+1})} \right) \right\} \mathrm{d}F(Z_t(i))
\end{multline}

$F(Z_t(i))$ is the distribution of idiosyncratic shocks across individuals and $\pi_{t+1}$ is the inflation rate from time $t$ to $t+1$.

One thing to note: downward nominal wage rigidities clearly affect workers who fall into the second line of \ref{eq:value_function}, those who can't lower their wage below $w$.
But the effect of downward nominal wage rigidities actually goes beyond that by lowering the future value of choosing too high a wage today, via the probability of now being able to lower it later.
This effect tends to compress the wages chosen by the $(1 - \lambda)$ unrestricted workers.

\section{Monetary Policy}
\label{sec:monetary_policy}

Not really implemented yet, but the central bank follows a normal Taylor Rule:

\begin{equation}
    \label{eq:taylor_rule}
    i_t = \frac{1 + \bar{pi}}{\beta} \left( \frac{Y_t}{\bar{Y}} \right)^{\varphi_Y} \left( \frac{1 + \pi_t}{1 + \bar{\pi}} \right)^{1 + \varphi_{\pi}} - 1
\end{equation}

where $\phi_Y, \phi_{\pi} > 0$ govern the strength of the central bank's reaction to deviations from its target level of inflation, $\bar{\pi}$ and the steady-state output level $\bar{Y}$.

\section{Equilibrium}
\label{sec:equilibrium}

With flexible wages, $\lambda = 0$, we can write the equilibrium level of output and employment in closed form.

\begin{equation}
    \label{eq:output_flexible}
    Y_t = L_t = \left( \frac{\eta - 1}{\eta} \right)^{\frac{\gamma}{1 + \gamma}} \left( \frac{1}{Z_t} \right)^{\frac{\gamma}{1 + \gamma}}
\end{equation}

where

\begin{align}
    Z_t =& \left( \int_{0}^{1}\! \left( \frac{1}{Z_t(i)} \right)^{\frac{\gamma(\eta - 1)}{\eta + \gamma}} \mathrm{d}i \right)^{-\frac{\eta + \gamma}{\gamma(\eta - 1)}}\\
        =& e^{-\frac{1}{2} \frac{\eta(1 + \gamma)}{\gamma + \eta}\sigma^2 }.
\end{align}

The last line follows from the specification that the idiosyncratic shocks $Z_t(i)$ are distributed lognoramlly.

For the rigid case, $\lambda > 0$, we need to keep track of which subset of individuals are constrained.
They can be broken into three groups.

The first group is those $(1 - \lambda)$ who are not constrained in the downward direction.

The second group is those who are not allowed to lower their wage, but draw an idiosyncratic shock such that they'd like to increase their wage anyway.

Finally, the third group is those who would like to lower their wage, but are not allowed to.

We can write the distribution of wages recursively:

\begin{equation}
    \label{eq:wage_distribution}
    G_t(w) = (1 - \lambda) F(z_t(w)) + \lambda G_{t-1}\left(w[1 + \pi_t]\right)F(z_t(w))
\end{equation}

Using the distribution of wages, we can solve for the output level with rigid wages:

\begin{equation}
    \label{eq:output_rigid}
    Y_t = L_t = \left(\frac{\eta - 1}{\eta} \right)^{\frac{\gamma}{1 + \gamma}}\left( \frac{1}{Z_t^*} \right)^{\frac{\gamma}{1 + \gamma}}
\end{equation}

where

\begin{equation}
    \label{eq:z_star}
    \begin{split}
    Z_t^* &= \Big\{(1 - \lambda) \int_{0}^{\infty} \! \left( \frac{1}{Z} \right)^{\frac{\gamma(\eta - 1)}{\eta + \gamma}} \left( \frac{\hat{w}_t(Z)}{w_t(Z)} \right)^{\eta - 1}\ \mathrm{d}F(Z) \\
          &+            \lambda  \int_{0}^{\infty} \! \left( \frac{1}{Z} \right)^{\frac{\gamma(\eta - 1)}{\eta + \gamma}} \left( \frac{\hat{w}_t(Z)}{w_t(Z)} \right)^{\eta - 1} G_{t-1}\left( w_t(Z)[1 + \pi_t] \right)                                                     \ \mathrm{d}F(Z)\\
          &+            \lambda  \int_{0}^{\infty} \! \left( \frac{1}{Z} \right)^{\frac{\gamma(\eta - 1)}{\eta + \gamma}} \left[ \int_{w_t(Z)}^{\infty} (1 + \pi_t)g_{t-1}\left( w[1 + \pi_t] \right) \left( \frac{\hat{w}_t(Z)}{w} \right)^{\eta - 1}\ \mathrm{d}w \right] \ \mathrm{d}F(Z)
            \Big\}^{-\frac{\eta + \gamma}{\gamma(\eta - 1)}}
    \end{split}
\end{equation}



\section{Steady State}
\label{sec:steady_state}


\bibliographystyle{apa}
\bibliography{citations}


\end{document}
