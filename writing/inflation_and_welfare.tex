%----------------------------------------------------------------------------------------
%    PAGE ADJUSTMENTS
%----------------------------------------------------------------------------------------

\documentclass[12pt,a4paper]{scrartcl}            % Article 12pt font for a4 paper while hiding links
\usepackage[margin=1.25in]{geometry}                          % Required to adjust margins

%----------------------------------------------------------------------------------------
%    TYPE SETTING PACKAGES
%----------------------------------------------------------------------------------------

\usepackage[english]{babel}                                % English language/hyphenation
\usepackage[utf8x]{inputenc}                               % Accept different input encodings
\usepackage{amsmath,amsfonts,amsthm,amssymb}               % Math packages to use equations
\usepackage{siunitx}                                       % Scientific units and numbering
\usepackage[usenames,dvipsnames,svgnames,table]{xcolor}    % Set color of text/background
\usepackage{titlesec}
\usepackage{datetime}

\titleformat{\section}
  {\Large\normalfont\scshape}{\thesection}{.5em}{}
\titleformat{\subsection}
  {\normalfont\scshape}{\thesubsection}{.5em}{}
% \linespread{1}                                           % Default line spacing size
% \usepackage{microtype}                                     % Improves spacing in the document
% \usepackage{setspace}                                      % Set line spacing dynamically
\usepackage{tocloft}                                       % List adjustments including ToC
\DeclareMathOperator*{\argmin}{arg\!\min}
\definecolor{purple}{HTML}{7A68A6}
\definecolor{blue}{HTML}{2020A1}
\definecolor{red}{HTML}{A60628}
%----------------------------------------------------------------------------------------
%    FIGURES
%----------------------------------------------------------------------------------------

\usepackage{graphicx}                                      % Required for the inclusion of images
\graphicspath{{../model/figures/}}                          % Specifies picture directory
\usepackage{float}                                         % Allows putting an [H] in \begin{figure}
\usepackage{wrapfig}                                       % Allows in-line images

\usepackage[ocgcolorlinks, colorlinks=true, citecolor=blue]{hyperref}     % References
\usepackage{cleveref}                                      % Better References
%\crefname{lstlisting}{listing}{listings}
%\Crefname{lstlisting}{Listing}{Listings}
\crefname{figure}{figure}{figures}
\Crefname{figure}{Figure}{Figures}

%----------------------------------------------------------------------------------------
%    INCLUDE CODE
%----------------------------------------------------------------------------------------

\usepackage{listings}                                      % Package so code looks pretty
\lstset{
language=Python,                                           % Choose the language
basicstyle=\footnotesize,                                  % The size of the fonts used
numbers=left,                                              % Where to put the line-numbers
numberstyle=\footnotesize,                                 % The size of the line-numbers
stepnumber=1,                                              % The step line-numbers
numbersep=5pt,                                             % How far the line-numbers are from the code
backgroundcolor=\color{white},                             % Choose the background color
showspaces=false,                                          % Show spaces adding partiular underscores
showstringspaces=false,                                    % Underline spaces within strings
showtabs=false,                                            % Show tabs within strings adding particular underscores
frame=single,                                              % Adds a frame around the code
tabsize=2,                                                 % Sets default tabsize to 2 spaces
captionpos=b,                                              % Sets the caption-position to bottom
breaklines=true,                                           % Sets automatic line breaking
breakatwhitespace=false,                                   % Sets if automatic breaks should only happen at whitespace
escapeinside={\%*}{*)}                                     % If you want to add a comment within your code
}

\usepackage[T1]{fontenc}
% \usepackage{inconsolata}


%----------------------------------------------------------------------------------------
%    EXTRAS
%----------------------------------------------------------------------------------------

% \usepackage{attachfile}                                    % Attach files to your document
% \usepackage{fancyhdr}                                      % Fancy Header
\usepackage{natbib}

\begin{document}

%----------------------------------------------------------------------------------------
%    COMMANDS
%----------------------------------------------------------------------------------------

% \renewcommand*\thesection{\arabic{section}}                % Renew section numbers
% \renewcommand{\labelenumi}{\alph{enumi}.}                  % Section ordered numbering
% \let\oldvec\vec                                            % Save the old vector style
% \renewcommand{\vec}[1]{\oldvec{\mathbf{#1}}}               % Set vectors to look like vectors

% \renewcommand{\contentsname}{Table of Contents}            % Make ToC actually say ToC
% \addtocontents{toc}{~\hfill\textbf{Page}\par}              % Add 'page' to top of ToC
% \renewcommand{\cftsecleader}{\cftdotfill{\cftdotsep}}      % Makes dots leading up to page number
% \setcounter{tocdepth}{3}                                   % Depth of ToC
% \setcounter{lofdepth}{3}                                   % Depth of LoF

% \pagestyle{plain}                                          % Fancy page style for headers
% \setlength{\headheight}{15pt}                              % Change header hieght
% \fancyhead[L,LO]{\fontsize{8}{10} \selectfont \firstmark}  % Adds header to left with section name
% \fancyhead[R,RO]{\fontsize{8}{10} \selectfont Right}       % Adds header to right
\definecolor{grey}{HTML}{cccccc}                           % The next 4 lines modifies the header (color)
% \renewcommand{\headrulewidth}{1px}
% \renewcommand{\headrule}{{\color{grey}%
% \hrule width\headwidth height\headrulewidth%
% \vskip-\headrulewidth}}

\numberwithin{equation}{section}                           % Number equations within sections
\numberwithin{figure}{section}                             % Number figures within sections
\numberwithin{table}{section}                              % Number tables within sections
\numberwithin{lstlisting}{section}                         % Number listings within sections

% \renewcommand{\sfdefault}{phv}                             % Change default font
% \renewcommand{\familydefault}{\sfdefault}                  % Use default font everywhere
\newcommand{\Lagr}{\mathcal{L}}
\newcommand{\tvect}[2]{%
  \ensuremath{\Bigl(\negthinspace\begin{smallmatrix}#1\\#2\end{smallmatrix}\Bigr)}}

\makeatletter
\def\blfootnote{\xdef\@thefnmark{}\@footnotetext}
\makeatother

\newdateformat{monthYear}{\monthname[\THEMONTH] \THEYEAR}

%----------------------------------------------------------------------------------------
%    TITLE PAGE
%----------------------------------------------------------------------------------------

% \begin{titlepage}
\title{Wage Rigidities and Inflation}
\author{Tom Augspurger}                               % via Seth Miers
\date{\monthYear\today}
\maketitle
%----------------------------------------------------------------------------------------
%    CONTENT
%----------------------------------------------------------------------------------------

\begin{abstract}
I use a simplified New Keynesian model to explore how downward nominal wage rigidities affect wage setting decisions.
Like previous research I find that wage rigidities distort the distribution of wage changes:
there are fewer wage cuts and more wage freezes than there would be if wages were flexible.
I also find that inflation helps to reduce some of the distortion resulting from wage rigidity by realigning wages that are too high.
Unlike previous research, I find that output is increasing in the degree of wage rigidity.

\end{abstract}

\section{Wage Rigidities}
\label{sec:wage_rigidities}

Wage rigidities and their importance have long been debated in economics.
One common argument is that wage rigidities force a tradeoff between output and price and wage stability.
This position is taken by \cite{erceg_henderson_levin_1999} and derivatives.
Others, like \cite{goodfriend_king_2001}, acknowledge the existence of rigid wages, but argue that in practice the rigidities will likely have little effect on macroeconomic outcomes.
% since they ``expect firms and workers to \emph{neutralize} allocative consequences of sticky nominal wages in the context of long-term relationships that predominate the labor market.''
I adopt a finding from recent research on wage rigidities: idiosyncratic shocks are likely necessary for wage rigidities to meaningfully affect aggregate output or employment.\footnote{
This is stressed in \cite{elsby_2009}, \cite{benigno_ricci_2011}, and \cite{daly_hobijn_2013}.}
I follow \cite{daly_hobijn_2013} and find that wage rigidities do affect aggregate output and the distribution of wage changes in a model with idiosyncratic shocks to labor preference.
Relative to the flexible wage benchmark, the distribution of wage changes with rigidities is asymmetric, with fewer wage cuts and more mass at no change.

Reviews of the relevant literature often start with a ``ritual piety'' \citep{tobin_1972} to Keynes's \emph{General Theory} before moving on to Tobin himself.
\cite{tobin_1972} opens by remarking on just how much the world changed in the forty some years between the 1930's, when he started as an economist, and his present day of the 1970's.
Despite those differences, ``[u]nemployment and inflation still preoccupy and perplex economists, statesmen, journalists, housewives, and everyone else.''
The intervening forty some years since Tobin's address has also seen remarkable change, but these same issues are still a preoccupation, though hopefully less perplexing.

On the empirical front, evidence for wage rigidities continues to mount.
\cite{mclaughlin_1994} and \cite{card_hyslop_1997} each used panel datasets to estimate the degree of downward wage rigidity and found that 15\%-20\% of nominal-yearly wage changes are negative, with a ``spike'' at no change.
\cite{akerlof_dickens_perry_1996} argued that the studies based on panel data understate the degree of downward wage rigidity, due to errors in reporting wage levels.
They augmented the earlier panel studies with surveys and Census data on manufacturing wages, both of which indicated more downward wage rigidity.
\cite{bewley_1999} conducted surveys with employers and labor market leaders and found that wage cuts were rare, almost only occurring when the firm was facing bankruptcy.
The most common reason cited for this resistance to nominal wage cuts was the negative effect wage cuts have on worker morale.

More recently, \cite{dickens_et_al_2006} applied a common methodology to measure wage rigidities to a host of countries and datasets.
While the variation across countries was substantial, a few common themes emerged.
Wage changes are clustered at the median and have many extreme values relative to the normal distribution.
Also, the distribution of wage changes is asymmetric; there are fewer wage cuts than raises, which implies nominal wage rigidity.
Finally, the wage changes tend to clump around the expected rate of inflation, which is interpreted as real wage rigidity.

I will take the existence of downward nominal wage rigidity as given.
And for the most part I will ignore the reasons for that wage rigidity.
Instead, focus is placed on the consequences of downward wage rigidities.\footnote{
    I should note that the evidence, causes, and consequences of wage rigidities may not be as separate as I make them out to be.
    For example, downward wage rigidities caused by fear of productivity loss following a wage cut might suggest a different modeling strategy than if the rigidities are caused by long-duration wage contracts.
    A different model could easily lead to different conclusions about the consequences and policy recommendations.
    Nevertheless I will take the narrower view here, and focus exclusively on the consequences.
}

One policy recommendation that often falls out of models with downward nominal wage rigidity is a higher inflation target.\footnote{
    See \cite{tobin_1972}, \cite{akerlof_dickens_perry_1996}, \cite{kim_ruge-murcia_2011}, \cite{coibon_gorodnichenko_wieland_2012}, among others.
}
New Keynesian models typically recommend an optimal inflation rate of zero, or even deflation equal to the nominal rate of interest \citep{schmitt-grohe_uribe_2010}.
However, if wages are rigid specifically in the downward direction, some of the concomitant distortion can be undone by inflation.
A wage that is ``too high'' can be brought back in line even without a nominal wage cut.

I use a a relatively simple model closely following \cite{daly_hobijn_2013} to examine the consequences of downward nominal wage rigidities.
Like them, I find the usual result that rigidities distort the distribution of wage changes, causing few wage cuts and a large spike of workers with no change in nominal wage.
However, I am unable to reproduce their results on how downward nominal wage rigidities affect the steady-state level of output.
They find what might be considered the more sensible result: that wage rigidities lower output, and inflation tends to offset some of that decline.
I find that wage rigidities tend to increase output (though not necessarily monotonically).

%----------------------------------------------------------------------------------------

\section{The Model}
\label{sec:the_model}

I follow \cite{daly_hobijn_2013} who presented a simplified version of \cite{benigno_ricci_2011}.
As a high-level overview, there is a continuum of individuals, making up the representative household, who sell their labor services to a perfectly competitive final good producer.
Individuals receive idiosyncratic shocks to their labor preference and face an exogenous impediment to lowering nominal wages.

\subsection{Production}
\label{sub:production}

There is a single producer who hires labor from members of the representative household and (linearly) transforms it into output, $Y_t$.

\begin{equation} \label{eq:agg_output}
    L_t = Y_t.
\end{equation}

Aggregate labor \(L_t\) is defined to be the integral over each individual's labor \(L_t(i)\), where the labor of one individual is an imperfect substitute for another:
%
\begin{equation} \label{eq:agg_labor}
    L_t = \left[ \int_0^1 L_t(i)^{\frac{\eta - 1}{\eta}} \mathrm{d}i \right]^{\frac{\eta}{\eta - 1}}
\end{equation}
%
where $\eta$ is the usual elasticity of substitution among the different types of labor.
% Since the different types of labor are not perfect substitutes (for $\eta < \infty$), the workers will charge a markup over the perfect competition equilibrium wage.

To simplify the aggregation process, I assume that there is a single final good producer who operates in a perfectly competitive market.
This final good producer's problem is to maximize profits (which will be zero in equilibrium) by choosing how much of each type of labor to hire:

\begin{equation} \label{eq:firms_problem}
    \max_{L_t(i)} P_t L_t - \int_0^1 W_t(i)L_t(i) \ \mathrm{d}i.
\end{equation}
%
taking individual wages $W_t(i)$ and the aggregate price level $P_t$ as given.

As shown in the appendix, solving this problem yields labor demand curves for each individual's labor:

\begin{equation}
    \label{eq:labor_demand}
    L_t(i) = \left( \frac{W_t(i)}{W_t} \right)^{-\eta}L_t = (w_{it})^{-\eta}L_t
\end{equation}
%
where $W_t$ is the aggregate wage index, defined to be

\begin{equation} \label{eq:wage_index}
    W_t = \left[\int_{0}^{1}\left(\frac{1}{W_t(i)}\right)^{\eta - 1} \mathrm{d}i \right]^{-\frac{1}{\eta - 1}}.
\end{equation}

Due to perfect competition we know that the aggregate price level must equal the aggregate wage index, $P_t = W_t$.  This implies that the real aggregate wage index is always equal to one and so
%
\begin{equation} \label{eq:real_wage}
    1 = \frac{W_t}{P_t} = \left[\int_{0}^{1} \left( \frac{P_t}{W_t(i)} \right)^{\eta - 1} \mathrm{d}i \right]^{-\frac{1}{\eta - 1} }\!\!\!\! = \ \left[\int_{0}^{1} \! w_{it}^{1 - \eta} \mathrm{d}i \right]^{ \frac{1}{1 - \eta} }
\end{equation}
%
where $w_{it}$ is the real wage charged by individual $i$.
We'll use \eqref{eq:real_wage} when solving for steady-state output;  the assumption of perfect competition, which is used to derive \eqref{eq:real_wage}, is important.

Aside from the lack of monopolistically-competitive intermediate producers, the production side of the economy is standard for this type of model.
The interesting choices in the model are made by the individuals, to whom we turn next.

\subsection{The household}
\label{sub:The household}

There is a representative household made up of a continuum of individuals.
As in \cite{erceg_henderson_levin_1999} individuals can share consumption risk perfectly, but make wage setting decisions on their own.
Once an individual has chosen a wage, he then supplies as much labor as is demanded by the firm at that wage.

The household earns utility from consumption and disutility from work:

\begin{equation} \label{eq:utility}
    \sum_{t=0}^{\infty} \beta^t \left\{\ln Y_t - \frac{\gamma}{\gamma + 1} \int_{0}^{1} Z_t(i)L_t(i)^{\frac{\gamma + 1}{\gamma}}\mathrm{d}i\right\}
\end{equation}
%
where $\gamma > 0$ is the elasticity of labor supply and $\beta \in (0, 1)$ is the discount factor.
The import bit of this utility function is that the labor preference shocks, $Z_t(i)$ are specific to each individual.
Note that I have already substituted in the market clearing condition that consumption equal output;
there's no capital or government spending in this model.

The utility function is maximized subject to the household budget constraint

\begin{equation}
    \label{eq:budget}
    P_t Y_t + A_t = \left(1 + i_{t-1}\right)A_{t-1} + \int_{0}^{1}\! W_t(i)L_t(i)\mathrm{d}i
\end{equation}
%
where $A_t$ is the level of nominal assets held by the household at time $t$ and $i_t$ is the nominal interest rate.
Since there is just the single household, $A_t$ will be zero in equilibrium.
Individuals also take their labor demand curve \eqref{eq:labor_demand} as given.
So, all else equal, individuals will work more the lower their own real wage or the higher aggregate output.

The household also takes the paths of aggregate wages, output, and the final good's price $\left\{W_t, L_t, P_t \right\}_{t=0}^{\infty}$ as given.

I'll assume that the idiosyncratic shocks to labor preference come from a lognormal distribution, i.i.d.\ across individuals and time:

\begin{equation}
    \label{eq:shock_dist}
    \ln(Z) \sim \mathcal{N}\left(-\frac{\sigma^2}{2}, \sigma\right)
\end{equation}
%
so that $\mathbb{E}(Z) = 1$.
This particular specification is chosen mainly out of convenience.
A shock specification that allows for persistence would not be unreasonable, but that path is not taken here.
% that said, \cite{daly_hobijn_2013} do manage a reasonable reproduction of reality with this simplified setup.


Finally, to the purpose at hand, I assume that nominal wages are rigid in the downward direction.
In the spirit of \cite{calvo_1983}, every period a randomly selected group of individuals of measure $\lambda \in \left(0, 1\right)$ are not allowed to \emph{lower} their nominal wages.
The other $\left(1 - \lambda \right)$ individuals are free to choose whatever wage they please.
\subsection{Wage Setting}
\label{sub:wage_setting}

As a point of comparison for the rigid case, we'll start with perfect wage flexibility~$(\lambda = 0)$.
Letting
\begin{equation}
    \label{eq:labor_part}
    \Omega( w_t(i); Z_t(i), L_t ) = w_{it}^{1 - \eta} - \frac{\gamma}{\gamma + 1}Z_t(i)\left[ w_{it}^{-\eta}L_t \right]^{\frac{\gamma}{1 + \gamma}}
\end{equation}
%
be the labor-related terms of an individual's problem, the optimal wage in the flexible case can can be attained by maximizing

\begin{equation}
    \label{eq:labor_opt}
    \mathbb{E}_t\left[\sum_{t=0}^{\infty}\beta^t \Omega( w_t(i); Z_t(i), L_t ) \right]
\end{equation}
%
subject to the budget constraint and the firm's labor demand schedule \eqref{eq:labor_demand}.
Solving this gives the optimal real wage schedule as a function of your idiosyncratic shock and aggregate output:

\begin{equation}
    \label{eq:flex}
    \hat{w}(Z_t(i); L_t) = \hat{w}_{it} = \left( \frac{\eta}{\eta - 1} \right)^{\frac{\gamma}{\eta + \gamma}}\left( Z_t(i) \right)^{\frac{\gamma}{\eta + \gamma}} L_t^{\frac{\gamma + 1}{\gamma + \eta}}.
\end{equation}
%
We will use this schedule when solving for equilibrium output.
% Consider going straight into equilibrium then looping back around to rigid case.

For the rigid case, $\lambda > 0$, our approach is slightly different.
Each day, our individual wakes up with his wage from yesterday.
He receives a labor preference shock, $Z_t(i)$, and finds out whether he is allowed to choose any nominal wage, or just a wage greater than or equal to yesterday's.
He then chooses a wage to optimize today's utility plus the discounted expected future value of holding that wage.
This can be written as a value function where the state variable is the real wage $w$:\footnote{
    See the appendix of \cite{daly_hobijn_2013} for properties of this type of value function.
}
\begin{multline}
    \label{eq:value_function}
    V_t(w) = (1 - \lambda) \int_{0}^{\infty} \max_{w_{it} \geq 0} \left\{ \Omega( w_t(i); Z_t(i), L_t ) + \beta V_{t+1}\left( \frac{w_{it}}{(1 + \pi_{t+1})} \right) \right\} \mathrm{d}F(Z_t(i)) \\
                + \lambda  \int_{0}^{\infty} \max_{w_{it} \geq w} \left\{ \Omega( w_t(i); Z_t(i), L_t ) + \beta V_{t+1}\left( \frac{w_{it}}{(1 + \pi_{t+1})} \right) \right\} \mathrm{d}F(Z_t(i))
\end{multline}

$F(\cdot)$ is the distribution of idiosyncratic shocks across individuals and $\pi_{t+1}$ is the inflation rate from time $t$ to $t+1$.
The value function gives us two useful functions: the (rigid) wage schedule $w_t(Z)$ and its inverse, the shock schedule $Z_t(w)$.
The wage schedule tells us what wage an unconstrained worker---one of the $(1 - \lambda)$---chooses when his idiosyncratic shock is $Z_t$.
We can show that $w_t(\cdot)$ is strictly increasing, and solve for its inverse $Z_t(w)$.

One thing to note: downward nominal wage rigidities clearly affect workers who fall into the second line of \eqref{eq:value_function}, those who can't lower their wage below $w$.
But the workers on the first line are also affected.
If a worker received a high shock and followed the \emph{flexible} wage schedule $\hat{w}(Z_t(i))$ they would choose a relatively high wage.
While that might be a good thing for today, it would ignore the possibility of receiving a low shock tomorrow, but not being able to lower the wage due to wage rigidity.
This effect tends to dampen the wage increases chosen by the $(1 - \lambda)$ unrestricted workers.
\cite{elsby_2009} investigates the importance of wage rigidity with an emphasis on this ``wage compression''.
Figure~\ref{fig:wage_schedules} gives an example of several wage schedules, the flexible wage schedule $\hat{w}_t(Z)$ and two rigid wage schedules for two levels of inflation.

\begin{center}
  \noindent \includegraphics[width=\linewidth]{wage_schedule_flex_rigid.pdf}  % TODO: Label your axes.
  \label{fig:wage_schedules}
\end{center}
%
Notice how the rigid wage schedules are always below the flexible wage schedule.
Apprehension of being unable to choose a lower wage in the future causes a lower wage choice today.
Also notice how higher inflation moves the wage schedule closer to the flexible case.


\section{Monetary Policy}
\label{sec:monetary_policy}

The central bank is assumed to follow a Taylor Rule to set the nominal interest rate $i_t$:

\begin{equation}
    \label{eq:taylor_rule}
    i_t = \frac{1 + \bar{\pi}}{\beta} \left( \frac{Y_t}{\bar{Y}} \right)^{\varphi_Y} \left( \frac{1 + \pi_t}{1 + \bar{\pi}} \right)^{1 + \varphi_{\pi}} - 1
\end{equation}
%
where $\phi_Y, \phi_{\pi} > 0$ govern the strength of the central bank's reaction to deviations from its target level of inflation, $\bar{\pi}$, and the steady-state output level $\bar{Y}$.
It's assumed that the zero lower bound on nominal rates will never bind.

\section{Equilibrium}
\label{sec:equilibrium}

With flexible wages, $\lambda = 0$, the equilibrium level of output and employment can be written in closed form:

\begin{equation}
    \label{eq:output_flexible}
    Y_t = L_t = \left( \frac{\eta - 1}{\eta} \right)^{\frac{\gamma}{1 + \gamma}} \left( \frac{1}{Z_t} \right)^{\frac{\gamma}{1 + \gamma}}
\end{equation}
%
where

\begin{equation}
    Z_t = \left( \int_{0}^{1}\! \left( \frac{1}{Z_t(i)} \right)^{\frac{\gamma(\eta - 1)}{\eta + \gamma}} \mathrm{d}i \right)^{-\frac{\eta + \gamma}{\gamma(\eta - 1)}} \!\!\! = e^{-\frac{1}{2} \frac{\eta(1 + \gamma)}{\gamma + \eta}\sigma^2 }.
\end{equation}
%
The second equality follows from the specification that the idiosyncratic shocks $Z_t(i)$ are distributed lognoramlly.

For the rigid case, $\lambda > 0$, it matters which subset of individuals are constrained.
The workers can be broken into three groups:
The first group is those $(1 - \lambda)$ individuals who are not constrained in the downward direction.
Since the restriction on who is allowed to lower his wage is entirely independent of the individual's current wage or how long he's had that wage, this group is dispersed equally over the range of wages.

The second group is those who are not allowed to lower their wage, but draw an idiosyncratic shock such that they'd like to increase their wage anyway.
Due to the ``wage compression'' effect discussed earlier, where workers choose lower wages than in the flexible case, the workers in groups one and two may end up working more than they would if wages were completely flexible.
In this model, anticipation of not being able to lower your wage tomorrow induces a lower wage today, which leads to more hours worked today.

Finally, the third group is those who would like to lower their wage, but are not allowed to.
This group will work fewer hours than they would under the flexible case.

Using $G_t(w)$ to denote the cumulative distribution function of wages at time $t$, we can write the distribution of wages recursively:

\begin{equation}
    \label{eq:wage_distribution}
    G_t(w) = (1 - \lambda) F(z_t(w)) + \lambda G_{t-1}\left(w[1 + \pi_t]\right)F(z_t(w))
\end{equation}

With the distribution of wages in hand, we can solve for the output level with wage rigidity:

\begin{equation}
    \label{eq:output_rigid}
    Y_t = L_t = \left(\frac{\eta - 1}{\eta} \right)^{\frac{\gamma}{1 + \gamma}}\left( \frac{1}{Z_t^*} \right)^{\frac{\gamma}{1 + \gamma}}
\end{equation}
%
where

\begin{equation}
    \label{eq:z_star}
    \begin{split}
    Z_t^* &= \Big\{(1 - \lambda) \int_{0}^{\infty} \! \left( \frac{1}{Z} \right)^{\frac{\gamma(\eta - 1)}{\eta + \gamma}} \left( \frac{\hat{w}_t(Z)}{w_t(Z)} \right)^{\eta - 1}\ \mathrm{d}F(Z) \\
          &+            \lambda  \int_{0}^{\infty} \! \left( \frac{1}{Z} \right)^{\frac{\gamma(\eta - 1)}{\eta + \gamma}} \left( \frac{\hat{w}_t(Z)}{w_t(Z)} \right)^{\eta - 1} G_{t-1}\left( w_t(Z)[1 + \pi_t] \right)                                                     \ \mathrm{d}F(Z)\\
          &+            \lambda  \int_{0}^{\infty} \! \left( \frac{1}{Z} \right)^{\frac{\gamma(\eta - 1)}{\eta + \gamma}} \left[ \int_{w_t(Z)}^{\infty} (1 + \pi_t)g_{t-1}\left( w[1 + \pi_t] \right) \left( \frac{\hat{w}_t(Z)}{w} \right)^{\eta - 1}\ \mathrm{d}w \right] \mathrm{d}F(Z)
            \Big\}^{-\frac{\eta + \gamma}{\gamma(\eta - 1)}}
    \end{split}
\end{equation}

Comparing to the flexible-wage case given in \eqref{eq:output_flexible}, the difference is in $Z_t$ and $Z_t^*$.
Let's consider two limiting cases.
First, as the degree of wage rigidity $\lambda$ goes to zero, only the first line of $Z_t^*$ is relevant.
When wages are flexible, the ratio $\frac{\hat{w}_t(Z)}{w_t(Z)}$ is one, since the wage schedule will be the flexible wage schedule $\hat{w}_t(Z)$.  % Holy batman awkward.
This leaves us with the definition of $Z_t$, the flexible case.

The other limiting case to consider is when inflation becomes large.
This case illustrates how inflation undoes some of the downward nominal wage rigidity.
With higher inflation fewer workers end up with a \emph{real} wage that is too high.
As inflation becomes large, the cumulative distribution function $G_t(w (1 + \pi))$ becomes one very quickly;
reflection on \eqref{eq:output_rigid} and \eqref{eq:z_star} shows how this leads to the flexible wage case.

The assumption of perfect competition in the final good market was used to derive \eqref{eq:output_rigid}.

\subsection{Steady State}
\label{sub:steady_state}

One comparison to be made between the flexible and rigid cases is the distribution of wage changes.
Under the flexible case, log normality of the idiosyncratic shocks implies that log wage changes should be normally distributed with a mean equal to the quarterly inflation rate.
With rigid wages, however, a different picture emerges.
The most prominent change is a higher spike at zero, no wage change.
This spike is taking mass from either side: there are fewer wage decreases \emph{and} fewer wage increases.
But the two tails are not sliced from equally.
The distribution is now asymmetric with fewer wage decreases than increases.
% TODO: Change to comparing w/ the normal.
\begin{center}
  \includegraphics[width=\linewidth]{pi0x005lambda0x452631578947_2_True.pdf}
  \label{fig:dist_2_periods}
\end{center}

If we look at the change over, say, four periods, the distribution of change looks more normal.

\begin{center}
  \includegraphics[width=\linewidth]{pi0x005lambda0x452631578947_4_True.pdf}
  \label{fig:dist_4_periods}
\end{center}


\subsection{Output}

Contrary to previous results, I find that wage rigidities are, for the most part, increasing in output.
In figure \ref{fig:output} I've shown the steady-state level of output over the range of rigidities simulated at a handful of inflation levels.

\begin{center}
  \includegraphics[width=\linewidth]{output_subsection.pdf}
  \label{fig:output}
\end{center}

This is somewhat puzzling.
The model does not expressly forbid higher rigidities leading to higher output;
earlier it was shown how those workers free to choose any wage chose one lower than they would have under the flexible case.
This does lead to more hours worked by those workers.
But the usual result---as in, say, \cite{daly_hobijn_2013}---is that this would offset but not overcome the loss of output due workers being stuck at too high a wage.
The only output level we have a closed form solution for is the flexible case, when $\lambda = 0$, which is given in \eqref{eq:output_flexible}.
In my simulations, as $\lambda$ converges to zero, the output levels do converge to the flexible output level, modulo estimation and floating-point error.
But the output level is increasing instead of decreasing in the degree of wage rigidity.

\section{Conclusion}

I used a simplified New Keynesian model whose interesting feature was idiosyncratic shocks to labor preference.
I found that output tended to be increasing in the degree of wage rigidity, though this ``claim'' rests on shaky ground.
The combination of idiosyncratic shocks with downward nominal wage rigidities, produced an asymmetrical distribution of wage changes.

What I have shown thus far should have been an intermediate step to novel topics and results.
It would, however, be unwise to continue down that path before understanding my contrarian results on output.

An especially glaring omission is how the simulated distribution of wage changes compares to the actual distribution.
That work has been started, but the analysis was not finished for this paper.

\newpage
\bibliographystyle{apa}
\bibliography{citations}

\newpage
\appendix

\section{Derivations}

Setup the Household's problem as a Lagrangian:

\begin{equation}
    \label{eqa:houehold}
    \begin{split}
        \Lagr : \sum_{t=0}^{\infty} \beta^t \Big\{ \ln Y_t & - \frac{\gamma}{\gamma + 1} \int_{0}^{1} \! Z_t(i)L_t(i) \frac{\gamma + 1}{\gamma} \mathrm{d}i \\
                                                           & - \mu_t \left[ (1 + i_{t-1})A_{t-1} + \int_{0}^{1} \! W_t(i)L_t(i) \mathrm{d}i  - A_t - P_tY_t\right] \Big\}
    \end{split}
\end{equation}

where $\mu_t$ is the Lagrangian multiplier on the budget constraint.
From this we get the first order conditions:

\begin{equation}
\begin{aligned}
    \label{eqa:focs}
    \frac{\partial \Lagr}{\partial Y} & :   \ \ \ \ \ \ \ \ \  \frac{\beta^t}{Y_t} = \mu_t \beta^t P_t \Rightarrow \mu_t = \frac{1}{P_t Y_t} \\
    \frac{\partial \Lagr}{\partial A_t} & : \ \ \ \ \ \ \ \ \  \mu_t = (1 + i_t) \mu_{t + 1} \beta \Rightarrow \frac{1}{Y_t} = \beta (1 + r_t) \frac{1}{Y_{t+1}}
\end{aligned}
\end{equation}

where $r_t$ is the real wage, $\frac{1 + i_t}{1 + \pi_{t+1}} - 1$.

The firm's problem is typical of the Dixit-Stiglitz framework:

\begin{equation}
    \label{eqa:firms_problem}
    \begin{aligned}
        & \min
        & & \int_{0}^{1} \! W_t(i)L_t(i) \mathrm{d}i \\
        & \text{subject to}
        & & Y_t \geq Y
    \end{aligned}
\end{equation}

where $Y$ is some minimum level of output.  The first order condition gives

\begin{equation}
    \label{eqa:firms_foc}
    \begin{aligned}
        W_t(i) &= \mu_t L_t(i)^{-1 / \eta} \left[ \int_{0}^{1} \! L_t(i)^{\frac{\eta - 1}{\eta}} \mathrm{d}i \right]^{\frac{\eta}{\eta - 1}}\\
               &= \mu_t L_t(i)^{-1 / \eta} L_t^{1 / \eta}.
    \end{aligned}
\end{equation}

Integrate over $i$ and raise to the $1 - \eta$:

\begin{equation}
    \label{eqa:firms_foc2}
    \begin{aligned}
        \int_{0}^{1} \! W_t(i)^{1 - \eta} \mathrm{d}i &= \mu_t^{1 - \eta} \int_{0}^{1} \! L_t(i)^{-\frac{1 + \eta}{\eta}} \mathrm{d}i L_t^{\frac{1 - \eta}{\eta}}\\
                                                      &= \mu_t^{1 - \eta}
    \end{aligned}
\end{equation}

If we let $W = \mu_t = \left[ \int_{0}^{1} \! W_t(i)^{1 - \eta} \mathrm{d}i \right]^{\frac{1}{1 - \eta}}$ be the wage index, we find that:

\begin{equation}
    \label{eqa:firms_foc3}
    L_t(i) = \left(\frac{W_t(i)}{\mu_t L_t(i)}\right)^{-\eta} = L_t \left( \frac{W_t(i)}{W_t} \right)^{-\eta}
\end{equation}

which is the firm's labor demand schedule, \eqref{eq:labor_demand}.
\section{Numerical Method}
\label{sec:numerical_method}

Parts of my numerical method follows \cite{daly_hobijn_2013}.
Since we are interested in the interaction of inflation and wage rigidities, I construct a grid of $\pi$, $\lambda$ pairs.
Then, given a level of output, I iterate on the value function in \eqref{eq:value_function} until convergence.
I use a mixture of linear and cubic splines to estimate the value function over the range of wages.
From the value function we get the wage schedule $w_(Z)$ and shock schedule, $Z_t(w)$, which are also estimated via cubic splines.

With the wage schedule in hand, I estimate an empirical cumulative distribution function for the wage distribution $G_t(w)$ by sampling from the steady state equations.
We obtain the histograms for steady-state wage changes by sampling a large number of individuals.

We can use the wage schedule and wage distribution to estimate \eqref{eq:output_rigid}.
Finally, we ensure that the level of output implied by \eqref{eq:output_rigid} is consistent with the level of output we started with.

See the documents described in Appendix \ref{sec:reproducability} for more information.

\section{Reproducibility}
\label{sec:reproducability}

Reproducibility matters.
Without getting mired in the, ``is economics a science?'' muddle, we can at least act scientifically.
To that end, all source files are available at this paper's \href{https://github.com/tomAugspurger/dnwr-zlb}{Github Repository}.
See \href{https://github.com/TomAugspurger/dnwr-zlb/blob/master/README.md}{\texttt{README.MD}} for a summary and installation instructions and \href{https://github.com/TomAugspurger/dnwr-zlb/blob/master/REQUIREMENTS.TXT}{\texttt{REQUIREMENTS.TXT}} for dependencies.
The data files themselves are not hosted at that webpage, but scripts to download them are included.

\end{document}
