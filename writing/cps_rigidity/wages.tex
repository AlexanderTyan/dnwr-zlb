\documentclass[11pt]{article}
\usepackage{amssymb}
\usepackage{amsmath}
\usepackage{booktabs}
\usepackage{graphicx}
\usepackage[usenames,dvipsnames,svgnames,table]{xcolor}    % Set color of text/background

\definecolor{purple}{HTML}{7A68A6}
\definecolor{blue}{HTML}{2020A1}
\definecolor{red}{HTML}{A60628}

\usepackage[colorlinks=true, citecolor=blue]{hyperref}     % References
\usepackage{cleveref}                                      % Better References
\usepackage{graphicx}                                      % Required for the inclusion of images
\usepackage{natbib}
\usepackage[font=small,labelfont=bf]{caption}

\title{Wage Rigidity in the CPS}
\author{Tom Augspurger}
\date{\today}
\begin{document}
\maketitle

\section{Intro}
\label{sec:intro}

The degree of rigidity in wages has interested economists for a while now.
I provide some evidence for the ongoing discussion.
I look at the rigidity of wages for newly hired workers and for workers in ongoing relationships separately.
The wages of workers in ongoing relationships are rigid, while the wages of newly hired workers are more flexible.

After \cite{shimer_2005} pointed out the inability of standard labor search and matching models to generate the observed fluctuations in unemployment and vacancies, wage rigidity was offered as a potential explanation.
For example, \cite{hall_milgrom_2008} posit a different bargaining structure between employers and workers that results in more rigid wages and generates more employment volatility than the standard model.

It has been used to explain employment fluctuations over the business cycle.
Recently, researchers have used the Consumer Population Survey (CPS) to measure wage flexibility.

\section{Data}
\label{sec:section_name}

A more thorough description of the data and the methods use to acquire and filter it can be found in the appendix.
Here, I will describe the layout of the CPS and what makes it useful for the problem at hand.

The CPS was established in 1940 and (since July 2001) interviews about 60,000 households per month.
A selected household is interviewed for four consecutive months, exits the survey for the next
eight months, and then returns to be surveyed for a final four months.
In total, a household is interviewed for eight months, spread over a twelve month period.

Certain questions are asked only of those households in months 4 and 8 (referred to as the Outgoing Rotation Group).
Importantly for this project, earnings questions are among these.
I constructed a panel matching individuals across their (potentially) 8 months in the survey.
The variables of interest include demographic variables such as age, race, education level, marital status, and gender;
employment variables such as labor force status (employed, unemployed, not in labor force), whether they remained at the same job, and hours worked;
and earnings variables, which are only available in survey months 4 and 8.

The CPS is attractive because of its long run (with micro-data back to the 1960s), large sample sizes, and broad suite of questions.
A drawback of the CPS is that it is not necessarily designed to be used longitudinally.
This makes matching individuals across time somewhat difficult (see \cite{madrian1999note} for a note on these topics).

\section{Search and Matching Models}
\label{sec:search_and_matching_models}

\cite{hall_milgrom_2008} showed that in the standard labor search and matching model, where the average product of labor is \emph{always} equal to its marginal product, unemployment volatility is one in the same as wage stickiness.
\cite{pissarides_2009} adjusts the standard labor searching and matching model in response to \cite{shimer_2005}'s and others' critiques.
He distinguishes the wage rigidity for workers in ongoing jobs from the wage rigidity for newly-hired workers.
He offers some evidence for the wages in ongoing jobs being especially rigid, while for new hires wages are quite flexible.
Here, wage rigidity or flexibility means how much the wage changes in response to a change in some measure of economic activity (e.g. labor productivity or the unemployment rate).

\section{Statistical Method}
\label{sec:statistical_method}

I use the method developed in \cite{haefke_sonntag_vanRens_2013} to estimate the degree of wage rigidity.
This method is designed to work with the structure of the CPS survey;
it allows for an estimate of wage rigidity without needing to observe an employed worker at regular individuals over time, which the CPS does not do.

Before describing their method, I will lay out how wage rigidity has typically been estimated.
The goal is to estimate some equation of the form:

\begin{equation}
    \label{eq:individual_wage}
    \ln w_{it} = \boldsymbol{x}_i^{T} \boldsymbol{\beta} + \ln \hat{w}_{it}
\end{equation}

where $w_{it}$ is the wage of an individual $i$ at time $t$, $\boldsymbol{x}_i$ is a vector of individual deterministic characteristics (such as age, gender, or race). $\hat{w}_{it}$ is the residual wage from this regression.

In the past, researchers would use longitudinal surveys and take the first difference of wages for a worker employed at two consecutive time periods (see, for example \citep{bils_1985}).
First differences are appropriate for two reasons.
First, it is possible that wage growth and measures of economic activity, like labor productivity, are cointegrated.
Second, first differences will remove individual heterogeneity.

This method has the negative effect of limiting the sample to only those workers employed in two consecutive periods;
you can't difference individual $i$'s wage if he didn't have a wage in the previous period.
As discussed in section \autoref{sec:search_and_matching_models}, the effects of wage rigidity among newly hired workers may differ from those of workers in ongoing relationships.

Finally getting to the method in \cite{haefke_sonntag_vanRens_2013}, the idea is to construct a wage index that controls for fluctuations in observable worker characteristics,
which fluctuate over time and the business cycle.
The wage index can then be regressed on a measure economic activity (labor productivity, or the unemployment rate, for example), to asses the degree of wage flexibility.
Since this method does not require taking the first difference of an individual's wage between two consecutive periods, newly hired workers can be included in the sample.
This in turn allows the CPS to be used.

As suggested by \cite{pissarides_2009}, currently employed workers at any date can be classified according to whether they were previously employed or are newly hired.
Following \cite{haefke_sonntag_vanRens_2013}, currently employed workers are classified as newly hired if they were either unemployed or not in the labor force in any of the preceding three months.

With this we can estimate the degree of wage rigidity for each group of workers:

\begin{equation}
    \label{eq:second_stage}
    \Delta \ln \hat{w}_{jt} = \alpha_j + \eta_j \Delta \ln y_t + \epsilon_{jt}
\end{equation}

where $j$ indicates the group of workers (previously employed or newly hired) and $y_t$ is taken to be the real productivity of labor.
With this setup, $\eta$ is the elasticity of wages with respect to productivity and has the convenient interpretation as the degree of wage rigidity.
As mentioned earlier, this equation is estimated in first differences;
but, thanks to using the wage index, the sample still includes newly hired workers.

\section{Results}
\label{sec:results}

I estimate \autoref{eq:second_stage} to produce the results in \autoref{tab:elasticity_estimates}.
Log real labor output per hour used as $y_t$
A value of 1 represents perfect wage flexibility, wages move 1 for 1 with productivity, while a value of 0 represents perfect rigidity.
The monthly data from the CPS are resampled at a quarterly frequency to match the frequency of the exogenous variable $y_t$.
Quarterly seasonal dummies are also included in the regression.
The key parameter is $\eta_j$, the estimate for the wage elasticity for either previously employed or non-employed workers.

The estimate of the wage elasticity for previously employed workers is nearly zero, suggesting that their wages are quite rigid.
For newly hired workers, the estimate is higher (0.33) suggesting more flexibility.
However, the standard error for the newly higher worker's elasticity (0.232) is also higher, partially reflecting the smaller sample size.

My estimates for the wage elasticity are smaller than those reported in \cite{haefke_sonntag_vanRens_2013},  who find values of 0.24 for all workers and 0.79 for new hires.
Our samples cover different time periods, so the estimates needn't match exactly, but the difference is large enough to merit some concern.
As reported in the appendix, their sample demographics information differs from mine;
it's possible that the differences are due to different matching criteria and filtering methods.

\begin{table}
    \centering
    \begin{tabular}{lccc} \toprule
        \textsc{Group}               & $\hat{\eta}_j$ & Standard Error & Observations\\
        \textsc{Newly hired}         & 0.3332       & 0.232            & 68,973       \\
        \textsc{Previously employed} & 0.0538       & 0.048            & 1,447,530     \\
        \textsc{All workers}         & 0.0702       & 0.050            & 1,516,503     \\ \bottomrule
    \end{tabular}
    \caption{
                Results from estimation of \autoref{eq:second_stage}.
                Workers are grouped according to their employment history.
                A worker is counted as \textsc{newly hired} if he was either unemployed
                or not in the labor force in any three months prior to his wage
                being observed. $\hat{\eta}_j$ is the estimate for the elasticity of the wage with respect to real labor productivity.
                The number of observations is the number of workers in that group.
            }
    \label{tab:elasticity_estimates}
\end{table}

\subsection{By Industry}

It's widely known that the behavior of durable and nondurable goods differs over the business cycle.
This suggests a similar analysis as above, but with workers grouped by their sector along with their employment history.
These results are presented in \autoref{tab:elasticity_estimates_sector}.

\begin{table}
    \centering
    \begin{tabular}{llrrr} \toprule
       \textsc{Sector}     & History              & $\hat{\eta_j}$ & Standard Error   & Observations\\
       \textsc{Durable}    & Newly hired          & -0.1410         & 0.470            & 4,705       \\
                  & Previously employed  & -0.0114         & 0.080            & 124,311     \\
                  & All workers          & -0.0058         & 0.077            & 129,016     \\
       \textsc{Nondurable} & Newly hired          & 0.9766          & 1.137            & 2,465       \\
                  & Previously employed  & 0.1156          & 0.198            & 63,148      \\
                  & All workers          & 0.1267          & 0.202            & 65,613      \\ \bottomrule
    \end{tabular}
    \caption{
                Similar results to \autoref{tab:elasticity_estimates}, but with workers grouped by major sector as well. Sectors are defined as in the BLS's Major Sector Productivity and Costs (see the \autoref{sec:appendix}).
            }
    \label{tab:elasticity_estimates_sector}
\end{table}

The estimates of wage rigidity for workers in durables industries are uniformly lower than for workers in nondurables.
Newly hired workers in nondurables industries have the highest estimated elasticity (.9766), but also the highest standard error (1.137) and the smallest sample size (just 2,465).

\section{Conclusion}

I used data from the CPS to estimate the degree of wage rigidity for various groups of workers.

\newpage
\bibliographystyle{apa}
\bibliography{citations}

\appendix

\section{Appendix}
\label{sec:appendix}

All source files are available at \href{https://github.com/tomAugspurger/dnwr-zlb}{https://github.com/tomAugspurger/dnwr-zlb}.

The CPS was not originally designed for longitudinal use.
Over the years, some variables have been added to facilitate the matching of individuals across interview months.

Matched data on the Outgoing Rotation Groups (ORG) can be fetched from the NBER website at http://www.nber.org/morg/annual/.
The included \texttt{morg\_downloader.py} file will do this.
We need to supplement the ORG data with monthly employment histories.
These are available from the raw monthly files.
Run \texttt{monthly\_data\_downloader.py} to gather the data dictionaries and monthly files.

All CPS variable names referred to are set in \texttt{monospace type} and refer to the January 2013 names.

An employment history for each worker can be constructed from the raw, unmatched, monthly data files themselves.
Because the CPS is not designed as a longitudinal survey, the unique household identifier in a given month does not necessarily identify the same household in the next month.
To be considered a valid match the following conditions must be met:

\begin{itemize}
    \item the individuals unique identifier (\texttt{HRHHID, HRHHID2, PULINENO}) must match
    \item the difference in age (\texttt{PRTAGE}) must be between -1 and 3 (to allow for aging measurement error)
    \item the reported sex (\texttt{PESEX}) must match
\end{itemize}

Ideally, each worker's employment status at each of the preceding 3 months is known.
If the employment status during any of the previous three months is unknown, the individual is dropped.
A worker is classified as coming from employment if they were employed in all of the previous three months (as well as this month).
A worker is classified as coming from unemployment if they were either unemployed or non-employed in \emph{any} of the preceding three months.
They must be employed this month for a wage to be reported.
The employment status is given by the labor force status recode (\texttt{PEMLR}).

As in \cite{haefke_sonntag_vanRens_2013}, if a worker reports his hours as variable (\texttt{PEHRUSL1} = -4), then his actual hours last month are used (\texttt{PEHRACT1}).

Productivity is measured by taken from the BLS's Major Sector Productivity and Costs program.
The series used are \texttt{PRS85006093} (Nonfarm Business Labor Output per Hour, base year 2009),
\textttt{PRS31006093} (same, but for just Durables), and \texttt{PRS32006093} (for nondurables).

Both wages and productivity measures are deflated by the BLS's implicit price deflator, \texttt{PRS85006143.
The productivity measures and deflator can be downloaded with \texttt{bls.py}.

Industry codes are taken from \texttt{PRDTIND1}.
The codes are broken into durable and nondurble at \href{http://www.bls.gov/news.release/prod2.tn.htm}{http://www.bls.gov/news.release/prod2.tn.htm}.
The following are durable industries:

\begin{center}
\begin{tabular}{lc} \toprule
NAICS Name                                      & BLS Code \\ \midrule
wood product manufacturing                      & 11       \\
nonmetallic mineral product manufacturing       & 5        \\
primary metal manufacturing                     & 6        \\
fabricated metal product manufacturing          & 6        \\
machinery manufacturing                         & 7        \\
computer and electronic product manufacturing   & 8        \\
electrical equipment and appliance manufacturing& 9        \\
transportation equipment manufacturing          & 10       \\
furniture and related product manufacturing     & 12       \\
miscellaneous manufacturing.                    & 13       \\
\end{tabular}
\end{center}
and nondurables:

\begin{center}
\begin{tabular}{lc} \toprule
NAICE Name                                  & BLS Code \\ \midrule
food manufacturing                          &  14      \\
beverage and tobacco product manufacturing  &  15      \\
textile mills                               &  16      \\
textile product mills                       &  16      \\
apparel manufacturing                       &  16      \\
leather and allied product manufacturing    &  16      \\
paper manufacturing                         &  17      \\
printing and related support activities     &  17      \\
petroleum and coal products manufacturing   &  18      \\
chemical manufacturing                      &  19      \\
plastics and rubber products manufacturing  &  20      \\
\end{tabular}
\end{center}

\begin{center}
  \noindent \includegraphics[width=1.25\linewidth]{demographics_ts.pdf}
  \label{fig:demographics_ts}
\end{center}

\begin{center}
  \noindent \includegraphics[width=1.25\linewidth]{wages_ts.pdf}
  \label{fig:earnings_ts}
\end{center}



\section{Conclusion}
\label{sec:conclusion}


\end{document}
