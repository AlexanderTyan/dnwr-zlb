\documentclass[11pt]{article}
\usepackage{amssymb}
\usepackage{amsmath}
\usepackage{booktabs}
\usepackage{graphicx}
\usepackage[usenames,dvipsnames,svgnames,table]{xcolor}    % Set color of text/background
% \usepackage{ctable}

\definecolor{purple}{HTML}{7A68A6}
\definecolor{blue}{HTML}{2020A1}
\definecolor{red}{HTML}{A60628}

\usepackage[colorlinks=true, citecolor=blue]{hyperref}     % References
\usepackage{cleveref}                                      % Better References
\usepackage{graphicx}                                      % Required for the inclusion of images
\usepackage{natbib}
\usepackage[font=small,labelfont=bf]{caption}

\DeclareMathOperator*{\argmin}{arg\!\min}
\DeclareMathOperator*{\argmax}{arg\!\max}


\title{Wage Rigidity and Employment History}
\author{Tom Augspurger}
\date{\today}
\begin{document}
\maketitle

%----------------------------------------------------------------------------------------

\abstract{
            I construct a rotating panel dataset from the monthly CPS micro-data files
            to measure wage rigidity. I find that the wages of workers in existing
            relationships are rigid, while the wages of newly-hired workers are more flexible.
            I also find that wages for workers in the nondurable goods sector are more flexible
            than in the durable goods sector.
}

%----------------------------------------------------------------------------------------
%----------------------------------------------------------------------------------------

\section{Introduction}
\label{sec:intro}

The degree of rigidity in wages has interested economists for a while now.
Wage rigidity of one form or another has been offered as justification for inflation (e.g. \cite{tobin_1972} or \cite{akerlof_dickens_perry_1996}) or as an explanation for the volatility of unemployment.
\cite{shimer_2005} noticed the inability of standard labor search and matching models to generate the observed fluctuations in unemployment and vacancies.
This led \cite{hall_milgrom_2008} to adjust the bargaining structure between employers and workers, and produce a search and matching model with rigid wages that generated more employment volatility than the standard model.

Attempts to measure wage rigidity go back at least to \cite{dunlop_1938}.
\cite{mclaughlin_1994} and \cite{card_hyslop_1997} each used panel datasets to estimate the degree of downward wage rigidity and found that 15\%-20\% of nominal-yearly wage changes are negative, with a ``spike'' at no change.
\cite{bewley_1999} conducted surveys with employers and labor market leaders and found that wage cuts were rare, almost only occurring when the firm was facing bankruptcy.
The most common reason cited for this resistance to nominal wage cuts was the negative effect wage cuts have on worker morale.
More recently, \cite{dickens_et_al_2006} applied a common wage rigidity measure to a host of countries and datasets.
While the variation across countries was substantial, a few common themes emerged.
Wage changes are clustered at the median and have many extreme values relative to the normal distribution.
Also, the distribution of wage changes is asymmetric; there are fewer wage cuts than raises, which implies nominal wage rigidity.
Finally, the wage changes tend to clump around the expected rate of inflation, which is interpreted as real wage rigidity.

I provide some additional evidence on wage rigidity for the continuing discussion.
A common theme in most of the empirical examples cited above is that researchers are measuring wage rigidity by differencing an individual's wage across two consecutive periods.
This has the unfortunate consequence of excluding newly-hired workers from the sample, which, as will be shown, may not be wise.
I look at the rigidity of wages for newly-hired workers and for workers in ongoing relationships separately.
The wages of workers in ongoing relationships are rigid (a wage elasticity with respect to productivity of 0.05), while the wages of newly-hired workers are more flexible (an elasticity of 0.33).

%----------------------------------------------------------------------------------------

\section{Data}
\label{sec:data}

A more thorough description of the data and the methods use to acquire and filter it can be found in the appendix.
Here, I will describe the layout of the CPS and what makes it useful for the problem at hand.

The CPS was established in 1940 and (since July 2001) interviews about 60,000 households per month.
A selected household is interviewed for four consecutive months, exits the survey for the next eight months, and then returns to be surveyed for a final four months.
In total, a household is interviewed for eight months, spread over a sixteen-month period.

Certain questions are asked only of those households in months 4 and 8 (referred to as the Outgoing Rotation Group).
Importantly for this project, earnings questions are among these.
Only observing the wages of a specific individual twice, separated by a 12-month gap, makes certain types of analyses infeasible.
In \autoref{sec:statistical_method} I describe one technique for working around this limitation.

I constructed a panel matching individuals across their (potentially) 8 months in the survey.
The variables of interest include demographic variables such as age, race, education level, marital status, gender, and work experience;
employment variables such as labor force status (employed, unemployed, not in labor force), whether they remained at the same job, industry worked in, and hours worked;
and earnings variables, which are only available in survey months 4 and 8.

The CPS is attractive because of its long run (with micro-data back to the 1960s), large sample sizes, and broad-ranging questions.
A drawback of the CPS is that it is not necessarily designed to be used longitudinally.
This makes matching individuals across time somewhat difficult (see \cite{madrian1999note} for a note on these topics).
I only use data from 1994 on, after the CPS redesign.
Also, due to a change in household identifiers in 1994, I'm only able to construct partial histories for most workers in 1995, so my sample starts in 1996.

%----------------------------------------------------------------------------------------

\section{Search and Matching Models}
\label{sec:search_and_matching_models}

\cite{hall_milgrom_2008} showed that in the standard labor search and matching model, where the average product of labor is \emph{always} equal to its marginal product, unemployment volatility is one in the same as wage stickiness.
\cite{pissarides_2009} adjusted the standard labor searching and matching model in response to \cite{shimer_2005}'s and others' critiques.
He distinguishes the wage rigidity for workers in ongoing jobs from the wage rigidity for newly-hired workers.
He offers some evidence for the wages in ongoing jobs being especially rigid, while for new hires wages are quite flexible.
Here, wage rigidity or flexibility means how much the wage changes in response to a change in some measure of economic activity (e.g. labor productivity or the unemployment rate).

Very briefly, I want to describe the standard labor search and matching model.
The details are not important for my results, but it will provide context.
For a more detailed layout, see (among many others) \cite{pissarides_2009}.

There is a pool of unemployed workers who match with a vacancy posted by an employer according the the matching technology, $m(u, v)$,
where $u$ is the pool of unemployed workers and $v$ is the pool of vacancies.
A key parameter in these models is the tightness ratio $\theta \equiv v/u$, the ratio of vacancies to unemployed.
Let $q(\theta) \equiv m(\theta, 1)$ denote the fill rate for each vacancy.
Everyone has linear utility, and unemployed workers receive an unemployment benefit worth $z$.
Employers can post vacancies at a flow cost of $c$ while the vacancy is unfilled.

If $V$ is defined to be the value of a new vacancy to the employer, then

\begin{equation}
    rV = -c + q(\theta)(J - V)
\end{equation}

where $J$ is the employer's value of an occupied job, which must satisfy:

\begin{equation}
    rJ = p - w - sJ
\end{equation}

where $r$ is the risk-free interest rate, $p$ is the flow value of output from a match, and $w$ is the wage.
By assumption, employers place zero value on a destroyed job.
The job creation condition is such that:

\begin{equation}
    V = 0 \iff \frac{p - w}{r + s} = \frac{c}{q(\theta)}
\end{equation}
\
That is, all available profits are exhausted.

At this point, a wage sharing rule must be specified.
In the usual setup, the wage splits the surplus created by the match in a fixed proportion.
Let $W$ be the worker's value from holding a job and $U$ be the worker's value of unemployment.
Then

\begin{equation}
    W - U = \beta (J + W - V - U)
\end{equation}

In \cite{pissarides_2009} the wage is determined by the Nash sharing rule:

\begin{equation}
    w = \argmax \left\{ (W - U)^\beta (J - V)^{1 - \beta} \right\}
\end{equation}

The wage equation satisfies

\begin{equation} \label{eq:wage_general}
    w = r U + \beta (p - r U - (r + s) V)
\end{equation}

Using the job creation condition, the value of unemployment must be such that

\begin{align*}
    rU &= z + f(\theta)(W - U)\\
       &= z + f(\theta)\frac{\beta}{1 - \beta}(J - V)\\
       &=z + \frac{\beta}{1 - \beta}c \theta
\end{align*}

where $f(\theta) \equiv m(1, v/u)$ is the transition rate of workers from unemployment to employment.
Using this in \autoref{eq:wage_general}, the wage equation now satisfies

\begin{equation}
    w = (1 - \beta) z + \beta ( p + c \theta )
\end{equation}

The interpretation here is that productivity shocks translate to wage changes in three ways.
First, there is a direct effect of productivity on the value of output from a match, $p$.
Second, an indirect effect on $p$ from the change in the reservation value for the worker.
And third, a similar indirect effect on $p$, but through the change in the reservation value for the firm.

The heart of \cite{shimer_2005}'s criticism was that in this setup, the wage is a weighted average of the worker's productivity and the value of unemployment.
If, say, productivity falls in a recession, then the worker's value in employment and in unemployment both fall by about the same amount.
The net result is not very much movement in unemployment.

%------------------------------------------------------------------------------

\section{Estimation Method}
\label{sec:statistical_method}

I use the method developed in \cite{haefke_sonntag_vanRens_2013} to estimate the degree of wage rigidity.
This method is designed to work with the structure of the CPS survey;
it allows for an estimate of wage rigidity without needing to observe an employed worker at regular individuals over time, which the CPS does not do.

Before describing their method, I will lay out how wage rigidity has typically been estimated.
The goal is to estimate some equation of the form:

\begin{equation}
    \label{eq:individual_wage}
    \ln w_{it} = \boldsymbol{x}_i^{T} \boldsymbol{\beta} + \ln \hat{w}_{it}
\end{equation}

where $w_{it}$ is the wage of an individual $i$ at time $t$, $\boldsymbol{x}_i$ is a vector of deterministic individual characteristics (such as age, gender, or race). $\hat{w}_{it}$ is the residual wage from this regression.

In the past, researchers would use longitudinal surveys and take the first difference of wages for a worker employed at two consecutive time periods (see, for example \cite{bils_1985}).
First differences are appropriate for two reasons.
First, it is possible that wage growth and measures of economic activity, like labor productivity, are cointegrated.
Second, first differences will remove individual heterogeneity.

This method has the negative effect of limiting the sample to only those workers employed in two consecutive periods;
you can't difference individual $i$'s wage if he didn't have a wage in the previous period.
As discussed in section \autoref{sec:search_and_matching_models}, the effects of wage rigidity among newly-hired workers may differ from those of workers in ongoing relationships, so they should be included in the sample.

Finally getting to the method in \cite{haefke_sonntag_vanRens_2013}, the idea is to construct a wage index that controls for fluctuations in observable worker characteristics, which fluctuate over time and the business cycle.
The wage index can then be regressed on a measure economic activity (labor productivity, or the unemployment rate, for example), to asses the degree of wage flexibility.
Since this method does not require taking the first difference of an individual's wage between two consecutive periods, newly-hired workers can be included in the sample.
Furthermore, this method allows the CPS to be used.
As discussed in \autoref{sec:data}, the CPS could not be used in a \cite{bils_1985} type setting since each individual is observed only twice, and those two observations are a year apart.

As suggested by \cite{pissarides_2009}, currently employed workers at any date can be classified according to whether they were previously employed or are newly hired.
Following \cite{haefke_sonntag_vanRens_2013}, currently employed workers are classified as newly hired if they were either unemployed or not in the labor force in any of the preceding three months.

With this we can estimate the degree of wage rigidity for each group of workers:

\begin{equation}
    \label{eq:second_stage}
    \Delta \ln \hat{w}_{jt} = \alpha_j + \eta_j \Delta \ln y_t + \epsilon_{jt}
\end{equation}

where $j$ indicates the group of workers (previously employed or newly hired) and $y_t$ is taken to be the real output per hour of labor.
With this setup, $\eta$ is the elasticity of wages with respect to productivity and has the convenient interpretation as the degree of wage rigidity.
As mentioned earlier, this equation is estimated in first differences (there is still the problem of cointegration);
but, thanks to using the wage index $\hat{w}_{jt}$, the sample still includes newly-hired workers.

%----------------------------------------------------------------------------------------

\section{Results}
\label{sec:results}

I estimate \autoref{eq:second_stage} to produce the results in \autoref{tab:elasticity_estimates}.
Log real labor output per hour is used as $y_t$.
A value of $\eta = 1$ represents perfect wage flexibility (wages move one-for-one with productivity) while a value of 0 represents perfect rigidity.
The monthly data from the CPS are resampled at a quarterly frequency to match the frequency of the independent variable $y_t$.
Quarterly seasonal dummies are also included in the regression.
The key parameter is $\eta_j$, the estimate for the wage elasticity for either previously employed or newly-hired workers.

The estimated elasticity for previously employed workers (0.0538) is nearly zero, suggesting that their wages are quite rigid.
For newly-hired workers, the estimate is higher (0.33) suggesting more flexibility.
However, the standard error for the newly-hired workers' elasticity (0.232) is also higher, partially reflecting the smaller sample size.

My estimates for the wage elasticity are smaller than those reported in \cite{haefke_sonntag_vanRens_2013},  who find values of 0.24 for all workers and 0.79 for new hires (they did not report result for just ongoing relationships).
Our samples cover different time periods, so the estimates needn't match exactly, but the difference is large enough to merit some concern.
As reported in the appendix, their sample demographics differ from mine;
it's possible that the differences are due to different matching criteria and filtering methods.

\begin{table}
    \centering
    \begin{tabular}{lccc} \toprule
        \textsc{Group}               & $\hat{\eta}_j$ & \textsc{Standard Error} & \textsc{Observations}\\
        \textsc{Newly hired}         & 0.3332         & 0.232                   & 68,973       \\
        \textsc{Previously employed} & 0.0538         & 0.048                   & 1,447,530     \\
        \textsc{All workers}         & 0.0702         & 0.050                   & 1,516,503     \\ \bottomrule
    \end{tabular}
    \caption{
                Results from estimation of \autoref{eq:second_stage}.
                Workers are grouped according to their employment history.
                A worker is counted as \textsc{newly hired} if he was either unemployed
                or not in the labor force in any three months prior to his wage
                being observed. $\hat{\eta}_j$ is the estimate for the elasticity of the wage with respect to real labor productivity.
                The number of observations is the number of workers in that group.
            }
    \label{tab:elasticity_estimates}
\end{table}

%----------------------------------------------------------------------------------------

\subsection{By Industry}

It's widely known that the behavior of durable and nondurable goods differs over the business cycle.
See, for example, \autoref{fig:productivity}.

\begin{figure}
    \begin{center}
      \includegraphics[width=\linewidth]{productivity.pdf}
    \end{center}
    \caption{
             Productivity (labor output per hour) by sector. Shaded areas indicate recessions.
            }
    \label{fig:productivity}
\end{figure}

This suggests a similar analysis as above, but with workers grouped by their sector along with their employment history.
These results are presented in \autoref{tab:elasticity_estimates_sector}.

\begin{table}
    \centering
    \begin{tabular}{llrrr} \toprule
       \textsc{Sector}     & \textsc{History}    & $\hat{\eta_j}$ & \textsc{Standard Error} & \textsc{Observations}\\
       \textsc{Durable}    & Newly hired         & -0.1410        & 0.470                   & 4,705       \\
                           & Previously employed & -0.0114        & 0.080                   & 124,311     \\
                           & All workers         & -0.0058        & 0.077                   & 129,016     \\
       \textsc{Nondurable} & Newly hired         & 0.9766         & 1.137                   & 2,465       \\
                           & Previously employed & 0.1156         & 0.198                   & 63,148      \\
                           & All workers         & 0.1267         & 0.202                   & 65,613      \\ \bottomrule
    \end{tabular}
    \caption{
                Similar results to \autoref{tab:elasticity_estimates}, but with workers grouped by major sector as well. Sectors are defined as in the BLS's Major Sector Productivity and Costs (see the \autoref{sec:appendix}).
            }
    \label{tab:elasticity_estimates_sector}
\end{table}

The estimates of wage rigidity for workers in durable goods industries are uniformly lower than for workers in nondurable goods industries.
Indeed, the point estimate for newly-hired workers is quite negative.
Newly-hired workers in nondurable goods industries have the highest estimated elasticity (.9766), but also the highest standard error (1.137) and the smallest sample size (just 2,465).

%----------------------------------------------------------------------------------------
\section{Conclusion}

I used data from the CPS to estimate the degree of wage rigidity for various groups of workers.
I found that the wages for new hires are more flexible than for workers in ongoing relationships.
This is in line with \cite{haefke_sonntag_vanRens_2013}, those my estimates indicate that wages are less flexible overall.
The results presented here support the claim in \cite{pissarides_2009} that models using wage rigidity to explain phenomena like unemployment volatility should be consistent with more wage flexibility among newly-hired workers than among those in ongoing employment.


\newpage
\bibliographystyle{apa}
\bibliography{citations}

%----------------------------------------------------------------------------------------

\appendix
\section{Appendix}
\label{sec:appendix}

All source files are available at \href{https://github.com/tomAugspurger/dnwr-zlb}{https://github.com/tomAugspurger/dnwr-zlb}.

The CPS was not originally designed for longitudinal use.
Over the years, some variables have been added to facilitate the matching of individuals across interview months.

Matched data on the Outgoing Rotation Groups (ORG) can be fetched from the NBER website at http://www.nber.org/morg/annual/.
The included \texttt{morg\_downloader.py} file will do this.
We need to supplement the ORG data with monthly employment histories.
These are available from the raw monthly files.
Run \texttt{monthly\_data\_downloader.py} to gather the data dictionaries and monthly files.

All CPS variable names referred to are set in \texttt{monospace type} and refer to the January 2013 names.

An employment history for each worker can be constructed from the raw, unmatched, monthly data files themselves.
Because the CPS is not designed as a longitudinal survey, the unique household identifier in a given month does not necessarily identify the same household in the next month.
To be considered a valid match the following conditions must be met:

\begin{itemize}
    \item the individuals unique identifier (\texttt{HRHHID, HRHHID2, PULINENO}) must match
    \item the difference in age (\texttt{PRTAGE}) must be between -1 and 3 (to allow for measurement error and aging)
    \item the reported sex (\texttt{PESEX}) must match
\end{itemize}

Ideally, each worker's employment status at each of the preceding 3 months is known.
If the employment status during any of the previous three months is unknown, the individual is dropped.
A worker is classified as coming from employment if they were employed in all of the previous three months (as well as this month).
A worker is classified as coming from unemployment if they were either unemployed or non-employed in \emph{any} of the preceding three months.
They must be employed this month for a wage to be reported.
The employment status is given by the labor force status recode (\texttt{PEMLR}).


As in \cite{haefke_sonntag_vanRens_2013}, if a worker reports his hours as variable (\texttt{PEHRUSL1} = -4), then his actual hours last month are used (\texttt{PEHRACT1}).

There are some extreme outliers in the hourly wage distribution.
The top and bottom .005\% of the wage distribution are trimmed.

Productivity is taken from the BLS's Major Sector Productivity and Costs program.
The series used are \texttt{PRS85006093} (Nonfarm Business Labor Output per Hour, base year 2009),
\texttt{PRS31006093} (same, but for just durables), and \texttt{PRS32006093} (for nondurables).

Both wages and productivity measures are deflated by the BLS's implicit price deflator, \texttt{PRS85006143}.
The productivity measures and deflator can be downloaded with \texttt{bls.py}.

Industry codes are taken from \texttt{PRDTIND1}.
The codes are broken into durable and nondurble at \href{http://www.bls.gov/news.release/prod2.tn.htm}{http://www.bls.gov/news.release/prod2.tn.htm}.
The following are durable industries:

\begin{center}
\begin{tabular}{lc} \toprule
NAICS Name                                      & BLS Code \\ \midrule
wood product manufacturing                      & 11       \\
nonmetallic mineral product manufacturing       & 5        \\
primary metal manufacturing                     & 6        \\
fabricated metal product manufacturing          & 6        \\
machinery manufacturing                         & 7        \\
computer and electronic product manufacturing   & 8        \\
electrical equipment and appliance manufacturing& 9        \\
transportation equipment manufacturing          & 10       \\
furniture and related product manufacturing     & 12       \\
miscellaneous manufacturing.                    & 13       \\
\end{tabular}
\end{center}
and nondurables:

\begin{center}
\begin{tabular}{lc} \toprule
NAICE Name                                  & BLS Code \\ \midrule
food manufacturing                          &  14      \\
beverage and tobacco product manufacturing  &  15      \\
textile mills                               &  16      \\
textile product mills                       &  16      \\
apparel manufacturing                       &  16      \\
leather and allied product manufacturing    &  16      \\
paper manufacturing                         &  17      \\
printing and related support activities     &  17      \\
petroleum and coal products manufacturing   &  18      \\
chemical manufacturing                      &  19      \\
plastics and rubber products manufacturing  &  20      \\
\end{tabular}
\end{center}

A worker's experience is defined to be his age minus his highest grade completed minus six.

\begin{figure}
\begin{center}
  \includegraphics[width=\linewidth]{demographics_ts.pdf}
  \caption{
            Demographics over time. Shaded areas indicate recessions as defined by the NBER.
            The education variable is a categorical variable with six values, similar to \cite{jaeger_1997}.
            The values are for No high-school, high-school or GED, some college, degree from a 4-year program,
            graduate degree from a 2 year program, and graduate degree from a longer than 2-year program.
            A person is counted as married if they are married and the spouse is present. Married and separated
            are counted as non-married.
          }
  \label{fig:demographics_ts}
\end{center}
\end{figure}

\begin{figure}
\begin{center}
  \includegraphics[width=\linewidth]{wages_ts.pdf}
  \caption{
            Average real wages over time for different subgroups. All wages are in 2009 dollars.
          }
  \label{fig:wages_ts}
\end{center}
\end{figure}

\begin{figure}
\begin{center}
  \includegraphics[width=\linewidth]{wages_by_age.pdf}
  \caption{
            Average real wages per hour across the age distribution. All wages are in 2009 dollars.
          }
  \label{fig:wages_by_age}
\end{center}
\end{figure}

\end{document}
